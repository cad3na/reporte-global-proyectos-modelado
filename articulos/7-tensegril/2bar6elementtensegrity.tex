%
% % Default to the notebook output style
%
%
%
%
% % Inherit from the specified cell style.
%
%
%
%
%
% \documentclass{article}
%
%
%
%     \usepackage{graphicx} % Used to insert images
%     \usepackage{adjustbox} % Used to constrain images to a maximum size
%     \usepackage{color} % Allow colors to be defined
%     \usepackage{enumerate} % Needed for markdown enumerations to work
%     \usepackage{geometry} % Used to adjust the document margins
%     \usepackage{amsmath} % Equations
%     \usepackage{amssymb} % Equations
%     \usepackage[mathletters]{ucs} % Extended unicode (utf-8) support
%     \usepackage[utf8x]{inputenc} % Allow utf-8 characters in the tex document
%     \usepackage{fancyvrb} % verbatim replacement that allows latex
%     \usepackage{grffile} % extends the file name processing of package graphics
%                          % to support a larger range
%     % The hyperref package gives us a pdf with properly built
%     % internal navigation ('pdf bookmarks' for the table of contents,
%     % internal cross-reference links, web links for URLs, etc.)
%     \usepackage{hyperref}
%     \usepackage{longtable} % longtable support required by pandoc >1.10
%     \usepackage{booktabs}  % table support for pandoc > 1.12.2
%
%
%
%
%     \definecolor{orange}{cmyk}{0,0.4,0.8,0.2}
%     \definecolor{darkorange}{rgb}{.71,0.21,0.01}
%     \definecolor{darkgreen}{rgb}{.12,.54,.11}
%     \definecolor{myteal}{rgb}{.26, .44, .56}
%     \definecolor{gray}{gray}{0.45}
%     \definecolor{lightgray}{gray}{.95}
%     \definecolor{mediumgray}{gray}{.8}
%     \definecolor{inputbackground}{rgb}{.95, .95, .85}
%     \definecolor{outputbackground}{rgb}{.95, .95, .95}
%     \definecolor{traceback}{rgb}{1, .95, .95}
%     % ansi colors
%     \definecolor{red}{rgb}{.6,0,0}
%     \definecolor{green}{rgb}{0,.65,0}
%     \definecolor{brown}{rgb}{0.6,0.6,0}
%     \definecolor{blue}{rgb}{0,.145,.698}
%     \definecolor{purple}{rgb}{.698,.145,.698}
%     \definecolor{cyan}{rgb}{0,.698,.698}
%     \definecolor{lightgray}{gray}{0.5}
%
%     % bright ansi colors
%     \definecolor{darkgray}{gray}{0.25}
%     \definecolor{lightred}{rgb}{1.0,0.39,0.28}
%     \definecolor{lightgreen}{rgb}{0.48,0.99,0.0}
%     \definecolor{lightblue}{rgb}{0.53,0.81,0.92}
%     \definecolor{lightpurple}{rgb}{0.87,0.63,0.87}
%     \definecolor{lightcyan}{rgb}{0.5,1.0,0.83}
%
%     % commands and environments needed by pandoc snippets
%     % extracted from the output of `pandoc -s`
%     \DefineVerbatimEnvironment{Highlighting}{Verbatim}{commandchars=\\\{\}}
%     % Add ',fontsize=\small' for more characters per line
%     \newenvironment{Shaded}{}{}
%     \newcommand{\KeywordTok}[1]{\textcolor[rgb]{0.00,0.44,0.13}{\textbf{{#1}}}}
%     \newcommand{\DataTypeTok}[1]{\textcolor[rgb]{0.56,0.13,0.00}{{#1}}}
%     \newcommand{\DecValTok}[1]{\textcolor[rgb]{0.25,0.63,0.44}{{#1}}}
%     \newcommand{\BaseNTok}[1]{\textcolor[rgb]{0.25,0.63,0.44}{{#1}}}
%     \newcommand{\FloatTok}[1]{\textcolor[rgb]{0.25,0.63,0.44}{{#1}}}
%     \newcommand{\CharTok}[1]{\textcolor[rgb]{0.25,0.44,0.63}{{#1}}}
%     \newcommand{\StringTok}[1]{\textcolor[rgb]{0.25,0.44,0.63}{{#1}}}
%     \newcommand{\CommentTok}[1]{\textcolor[rgb]{0.38,0.63,0.69}{\textit{{#1}}}}
%     \newcommand{\OtherTok}[1]{\textcolor[rgb]{0.00,0.44,0.13}{{#1}}}
%     \newcommand{\AlertTok}[1]{\textcolor[rgb]{1.00,0.00,0.00}{\textbf{{#1}}}}
%     \newcommand{\FunctionTok}[1]{\textcolor[rgb]{0.02,0.16,0.49}{{#1}}}
%     \newcommand{\RegionMarkerTok}[1]{{#1}}
%     \newcommand{\ErrorTok}[1]{\textcolor[rgb]{1.00,0.00,0.00}{\textbf{{#1}}}}
%     \newcommand{\NormalTok}[1]{{#1}}
%
%     % Define a nice break command that doesn't care if a line doesn't already
%     % exist.
%     \def\br{\hspace*{\fill} \\* }
%     % Math Jax compatability definitions
%     \def\gt{>}
%     \def\lt{<}
%     % Document parameters
%     \title{2 bar 6 element tensegrity}
%
%
%
%
%     % Pygments definitions
%
% \makeatletter
% \def\PY@reset{\let\PY@it=\relax \let\PY@bf=\relax%
%     \let\PY@ul=\relax \let\PY@tc=\relax%
%     \let\PY@bc=\relax \let\PY@ff=\relax}
% \def\PY@tok#1{\csname PY@tok@#1\endcsname}
% \def\PY@toks#1+{\ifx\relax#1\empty\else%
%     \PY@tok{#1}\expandafter\PY@toks\fi}
% \def\PY@do#1{\PY@bc{\PY@tc{\PY@ul{%
%     \PY@it{\PY@bf{\PY@ff{#1}}}}}}}
% \def\PY#1#2{\PY@reset\PY@toks#1+\relax+\PY@do{#2}}
%
% \expandafter\def\csname PY@tok@gd\endcsname{\def\PY@tc##1{\textcolor[rgb]{0.63,0.00,0.00}{##1}}}
% \expandafter\def\csname PY@tok@gu\endcsname{\let\PY@bf=\textbf\def\PY@tc##1{\textcolor[rgb]{0.50,0.00,0.50}{##1}}}
% \expandafter\def\csname PY@tok@gt\endcsname{\def\PY@tc##1{\textcolor[rgb]{0.00,0.27,0.87}{##1}}}
% \expandafter\def\csname PY@tok@gs\endcsname{\let\PY@bf=\textbf}
% \expandafter\def\csname PY@tok@gr\endcsname{\def\PY@tc##1{\textcolor[rgb]{1.00,0.00,0.00}{##1}}}
% \expandafter\def\csname PY@tok@cm\endcsname{\let\PY@it=\textit\def\PY@tc##1{\textcolor[rgb]{0.25,0.50,0.50}{##1}}}
% \expandafter\def\csname PY@tok@vg\endcsname{\def\PY@tc##1{\textcolor[rgb]{0.10,0.09,0.49}{##1}}}
% \expandafter\def\csname PY@tok@m\endcsname{\def\PY@tc##1{\textcolor[rgb]{0.40,0.40,0.40}{##1}}}
% \expandafter\def\csname PY@tok@mh\endcsname{\def\PY@tc##1{\textcolor[rgb]{0.40,0.40,0.40}{##1}}}
% \expandafter\def\csname PY@tok@go\endcsname{\def\PY@tc##1{\textcolor[rgb]{0.53,0.53,0.53}{##1}}}
% \expandafter\def\csname PY@tok@ge\endcsname{\let\PY@it=\textit}
% \expandafter\def\csname PY@tok@vc\endcsname{\def\PY@tc##1{\textcolor[rgb]{0.10,0.09,0.49}{##1}}}
% \expandafter\def\csname PY@tok@il\endcsname{\def\PY@tc##1{\textcolor[rgb]{0.40,0.40,0.40}{##1}}}
% \expandafter\def\csname PY@tok@cs\endcsname{\let\PY@it=\textit\def\PY@tc##1{\textcolor[rgb]{0.25,0.50,0.50}{##1}}}
% \expandafter\def\csname PY@tok@cp\endcsname{\def\PY@tc##1{\textcolor[rgb]{0.74,0.48,0.00}{##1}}}
% \expandafter\def\csname PY@tok@gi\endcsname{\def\PY@tc##1{\textcolor[rgb]{0.00,0.63,0.00}{##1}}}
% \expandafter\def\csname PY@tok@gh\endcsname{\let\PY@bf=\textbf\def\PY@tc##1{\textcolor[rgb]{0.00,0.00,0.50}{##1}}}
% \expandafter\def\csname PY@tok@ni\endcsname{\let\PY@bf=\textbf\def\PY@tc##1{\textcolor[rgb]{0.60,0.60,0.60}{##1}}}
% \expandafter\def\csname PY@tok@nl\endcsname{\def\PY@tc##1{\textcolor[rgb]{0.63,0.63,0.00}{##1}}}
% \expandafter\def\csname PY@tok@nn\endcsname{\let\PY@bf=\textbf\def\PY@tc##1{\textcolor[rgb]{0.00,0.00,1.00}{##1}}}
% \expandafter\def\csname PY@tok@no\endcsname{\def\PY@tc##1{\textcolor[rgb]{0.53,0.00,0.00}{##1}}}
% \expandafter\def\csname PY@tok@na\endcsname{\def\PY@tc##1{\textcolor[rgb]{0.49,0.56,0.16}{##1}}}
% \expandafter\def\csname PY@tok@nb\endcsname{\def\PY@tc##1{\textcolor[rgb]{0.00,0.50,0.00}{##1}}}
% \expandafter\def\csname PY@tok@nc\endcsname{\let\PY@bf=\textbf\def\PY@tc##1{\textcolor[rgb]{0.00,0.00,1.00}{##1}}}
% \expandafter\def\csname PY@tok@nd\endcsname{\def\PY@tc##1{\textcolor[rgb]{0.67,0.13,1.00}{##1}}}
% \expandafter\def\csname PY@tok@ne\endcsname{\let\PY@bf=\textbf\def\PY@tc##1{\textcolor[rgb]{0.82,0.25,0.23}{##1}}}
% \expandafter\def\csname PY@tok@nf\endcsname{\def\PY@tc##1{\textcolor[rgb]{0.00,0.00,1.00}{##1}}}
% \expandafter\def\csname PY@tok@si\endcsname{\let\PY@bf=\textbf\def\PY@tc##1{\textcolor[rgb]{0.73,0.40,0.53}{##1}}}
% \expandafter\def\csname PY@tok@s2\endcsname{\def\PY@tc##1{\textcolor[rgb]{0.73,0.13,0.13}{##1}}}
% \expandafter\def\csname PY@tok@vi\endcsname{\def\PY@tc##1{\textcolor[rgb]{0.10,0.09,0.49}{##1}}}
% \expandafter\def\csname PY@tok@nt\endcsname{\let\PY@bf=\textbf\def\PY@tc##1{\textcolor[rgb]{0.00,0.50,0.00}{##1}}}
% \expandafter\def\csname PY@tok@nv\endcsname{\def\PY@tc##1{\textcolor[rgb]{0.10,0.09,0.49}{##1}}}
% \expandafter\def\csname PY@tok@s1\endcsname{\def\PY@tc##1{\textcolor[rgb]{0.73,0.13,0.13}{##1}}}
% \expandafter\def\csname PY@tok@kd\endcsname{\let\PY@bf=\textbf\def\PY@tc##1{\textcolor[rgb]{0.00,0.50,0.00}{##1}}}
% \expandafter\def\csname PY@tok@sh\endcsname{\def\PY@tc##1{\textcolor[rgb]{0.73,0.13,0.13}{##1}}}
% \expandafter\def\csname PY@tok@sc\endcsname{\def\PY@tc##1{\textcolor[rgb]{0.73,0.13,0.13}{##1}}}
% \expandafter\def\csname PY@tok@sx\endcsname{\def\PY@tc##1{\textcolor[rgb]{0.00,0.50,0.00}{##1}}}
% \expandafter\def\csname PY@tok@bp\endcsname{\def\PY@tc##1{\textcolor[rgb]{0.00,0.50,0.00}{##1}}}
% \expandafter\def\csname PY@tok@c1\endcsname{\let\PY@it=\textit\def\PY@tc##1{\textcolor[rgb]{0.25,0.50,0.50}{##1}}}
% \expandafter\def\csname PY@tok@kc\endcsname{\let\PY@bf=\textbf\def\PY@tc##1{\textcolor[rgb]{0.00,0.50,0.00}{##1}}}
% \expandafter\def\csname PY@tok@c\endcsname{\let\PY@it=\textit\def\PY@tc##1{\textcolor[rgb]{0.25,0.50,0.50}{##1}}}
% \expandafter\def\csname PY@tok@mf\endcsname{\def\PY@tc##1{\textcolor[rgb]{0.40,0.40,0.40}{##1}}}
% \expandafter\def\csname PY@tok@err\endcsname{\def\PY@bc##1{\setlength{\fboxsep}{0pt}\fcolorbox[rgb]{1.00,0.00,0.00}{1,1,1}{\strut ##1}}}
% \expandafter\def\csname PY@tok@mb\endcsname{\def\PY@tc##1{\textcolor[rgb]{0.40,0.40,0.40}{##1}}}
% \expandafter\def\csname PY@tok@ss\endcsname{\def\PY@tc##1{\textcolor[rgb]{0.10,0.09,0.49}{##1}}}
% \expandafter\def\csname PY@tok@sr\endcsname{\def\PY@tc##1{\textcolor[rgb]{0.73,0.40,0.53}{##1}}}
% \expandafter\def\csname PY@tok@mo\endcsname{\def\PY@tc##1{\textcolor[rgb]{0.40,0.40,0.40}{##1}}}
% \expandafter\def\csname PY@tok@kn\endcsname{\let\PY@bf=\textbf\def\PY@tc##1{\textcolor[rgb]{0.00,0.50,0.00}{##1}}}
% \expandafter\def\csname PY@tok@mi\endcsname{\def\PY@tc##1{\textcolor[rgb]{0.40,0.40,0.40}{##1}}}
% \expandafter\def\csname PY@tok@gp\endcsname{\let\PY@bf=\textbf\def\PY@tc##1{\textcolor[rgb]{0.00,0.00,0.50}{##1}}}
% \expandafter\def\csname PY@tok@o\endcsname{\def\PY@tc##1{\textcolor[rgb]{0.40,0.40,0.40}{##1}}}
% \expandafter\def\csname PY@tok@kr\endcsname{\let\PY@bf=\textbf\def\PY@tc##1{\textcolor[rgb]{0.00,0.50,0.00}{##1}}}
% \expandafter\def\csname PY@tok@s\endcsname{\def\PY@tc##1{\textcolor[rgb]{0.73,0.13,0.13}{##1}}}
% \expandafter\def\csname PY@tok@kp\endcsname{\def\PY@tc##1{\textcolor[rgb]{0.00,0.50,0.00}{##1}}}
% \expandafter\def\csname PY@tok@w\endcsname{\def\PY@tc##1{\textcolor[rgb]{0.73,0.73,0.73}{##1}}}
% \expandafter\def\csname PY@tok@kt\endcsname{\def\PY@tc##1{\textcolor[rgb]{0.69,0.00,0.25}{##1}}}
% \expandafter\def\csname PY@tok@ow\endcsname{\let\PY@bf=\textbf\def\PY@tc##1{\textcolor[rgb]{0.67,0.13,1.00}{##1}}}
% \expandafter\def\csname PY@tok@sb\endcsname{\def\PY@tc##1{\textcolor[rgb]{0.73,0.13,0.13}{##1}}}
% \expandafter\def\csname PY@tok@k\endcsname{\let\PY@bf=\textbf\def\PY@tc##1{\textcolor[rgb]{0.00,0.50,0.00}{##1}}}
% \expandafter\def\csname PY@tok@se\endcsname{\let\PY@bf=\textbf\def\PY@tc##1{\textcolor[rgb]{0.73,0.40,0.13}{##1}}}
% \expandafter\def\csname PY@tok@sd\endcsname{\let\PY@it=\textit\def\PY@tc##1{\textcolor[rgb]{0.73,0.13,0.13}{##1}}}
%
% \def\PYZbs{\char`\\}
% \def\PYZus{\char`\_}
% \def\PYZob{\char`\{}
% \def\PYZcb{\char`\}}
% \def\PYZca{\char`\^}
% \def\PYZam{\char`\&}
% \def\PYZlt{\char`\<}
% \def\PYZgt{\char`\>}
% \def\PYZsh{\char`\#}
% \def\PYZpc{\char`\%}
% \def\PYZdl{\char`\$}
% \def\PYZhy{\char`\-}
% \def\PYZsq{\char`\'}
% \def\PYZdq{\char`\"}
% \def\PYZti{\char`\~}
% % for compatibility with earlier versions
% \def\PYZat{@}
% \def\PYZlb{[}
% \def\PYZrb{]}
% \makeatother
%
%
%     % Exact colors from NB
%     \definecolor{incolor}{rgb}{0.0, 0.0, 0.5}
%     \definecolor{outcolor}{rgb}{0.545, 0.0, 0.0}
%
%
%
%
%     % Prevent overflowing lines due to hard-to-break entities
%     \sloppy
%     % Setup hyperref package
%     \hypersetup{
%       breaklinks=true,  % so long urls are correctly broken across lines
%       colorlinks=true,
%       urlcolor=blue,
%       linkcolor=darkorange,
%       citecolor=darkgreen,
%       }
%     % Slightly bigger margins than the latex defaults
%
%     \geometry{verbose,tmargin=1in,bmargin=1in,lmargin=1in,rmargin=1in}
%
%
%
%     \begin{document}
%

    \maketitle





    \section{Stability analysis of a 2-bar and 6-element tensegrity}


    Given the following configuration, we want to know the internal forces
necessary for the system to be in equilibrium.

\begin{figure}[htbp]
\centering
\includegraphics{./imagenes/fig2_1.png}
\caption{Figure 2.1}
\end{figure}

We know that in a static structure the sum of the internal forces must
be equal to \(0\), so the first thing that we have to obtain is a \(F\)
that represents the sum of the internal forces.

    We will begin by importing the symbolic calculus library

    \begin{Verbatim}[commandchars=\\\{\}]
{\color{incolor}In [{\color{incolor}1}]:} \PY{k+kn}{from} \PY{n+nn}{sympy} \PY{k+kn}{import} \PY{n}{init\PYZus{}printing}\PY{p}{,} \PY{n}{var}\PY{p}{,} \PY{n}{Matrix}\PY{p}{,} \PY{n}{solve}
\end{Verbatim}

    \begin{Verbatim}[commandchars=\\\{\}]
{\color{incolor}In [{\color{incolor}2}]:} \PY{n}{init\PYZus{}printing}\PY{p}{(}\PY{p}{)}
\end{Verbatim}

    And initializing the symbolic variables that represent the internal
forces on the nodes.

    \begin{Verbatim}[commandchars=\\\{\}]
{\color{incolor}In [{\color{incolor}3}]:} \PY{n}{var}\PY{p}{(}\PY{l+s}{\PYZdq{}}\PY{l+s}{lambda:3}\PY{l+s}{\PYZdq{}}\PY{p}{)}
        \PY{n}{var}\PY{p}{(}\PY{l+s}{\PYZdq{}}\PY{l+s}{gamma:5}\PY{l+s}{\PYZdq{}}\PY{p}{)}\PY{p}{;}
\end{Verbatim}

    We define the vectors that describe the nodes:

    \begin{Verbatim}[commandchars=\\\{\}]
{\color{incolor}In [{\color{incolor}4}]:} \PY{n}{n1} \PY{o}{=} \PY{n}{Matrix}\PY{p}{(}\PY{p}{[}\PY{l+m+mi}{1}\PY{p}{,} \PY{l+m+mi}{0}\PY{p}{]}\PY{p}{)}
        \PY{n}{n2} \PY{o}{=} \PY{n}{Matrix}\PY{p}{(}\PY{p}{[}\PY{l+m+mi}{0}\PY{p}{,} \PY{l+m+mi}{1}\PY{p}{]}\PY{p}{)}
        \PY{n}{n3} \PY{o}{=} \PY{n}{Matrix}\PY{p}{(}\PY{p}{[}\PY{o}{\PYZhy{}}\PY{l+m+mi}{1}\PY{p}{,} \PY{l+m+mi}{0}\PY{p}{]}\PY{p}{)}
        \PY{n}{n4} \PY{o}{=} \PY{n}{Matrix}\PY{p}{(}\PY{p}{[}\PY{l+m+mi}{0}\PY{p}{,} \PY{o}{\PYZhy{}}\PY{l+m+mi}{1}\PY{p}{]}\PY{p}{)}

        \PY{n}{N} \PY{o}{=} \PY{p}{(}\PY{p}{(}\PY{n}{n1}\PY{o}{.}\PY{n}{row\PYZus{}join}\PY{p}{(}\PY{n}{n2}\PY{p}{)}\PY{p}{)}\PY{o}{.}\PY{n}{row\PYZus{}join}\PY{p}{(}\PY{n}{n3}\PY{p}{)}\PY{p}{)}\PY{o}{.}\PY{n}{row\PYZus{}join}\PY{p}{(}\PY{n}{n4}\PY{p}{)}
        \PY{n}{N}
\end{Verbatim}
\texttt{\color{outcolor}Out[{\color{outcolor}4}]:}


        \begin{equation*}
        \left[\begin{matrix}1 & 0 & -1 & 0\\0 & 1 & 0 & -1\end{matrix}\right]
        \end{equation*}



    So the vectors that describe the bars are:

    \begin{Verbatim}[commandchars=\\\{\}]
{\color{incolor}In [{\color{incolor}5}]:} \PY{n}{b1} \PY{o}{=} \PY{n}{n3} \PY{o}{\PYZhy{}} \PY{n}{n1}
        \PY{n}{b2} \PY{o}{=} \PY{n}{n4} \PY{o}{\PYZhy{}} \PY{n}{n2}

        \PY{n}{B} \PY{o}{=} \PY{n}{b1}\PY{o}{.}\PY{n}{row\PYZus{}join}\PY{p}{(}\PY{n}{b2}\PY{p}{)}
        \PY{n}{B}
\end{Verbatim}
\texttt{\color{outcolor}Out[{\color{outcolor}5}]:}


        \begin{equation*}
        \left[\begin{matrix}-2 & 0\\0 & -2\end{matrix}\right]
        \end{equation*}



    And the strings are characterized by:

    \begin{Verbatim}[commandchars=\\\{\}]
{\color{incolor}In [{\color{incolor}6}]:} \PY{n}{s1} \PY{o}{=} \PY{n}{n2} \PY{o}{\PYZhy{}} \PY{n}{n1}
        \PY{n}{s2} \PY{o}{=} \PY{n}{n3} \PY{o}{\PYZhy{}} \PY{n}{n2}
        \PY{n}{s3} \PY{o}{=} \PY{n}{n4} \PY{o}{\PYZhy{}} \PY{n}{n3}
        \PY{n}{s4} \PY{o}{=} \PY{n}{n1} \PY{o}{\PYZhy{}} \PY{n}{n4}

        \PY{n}{S} \PY{o}{=} \PY{p}{(}\PY{p}{(}\PY{n}{s1}\PY{o}{.}\PY{n}{row\PYZus{}join}\PY{p}{(}\PY{n}{s2}\PY{p}{)}\PY{p}{)}\PY{o}{.}\PY{n}{row\PYZus{}join}\PY{p}{(}\PY{n}{s3}\PY{p}{)}\PY{p}{)}\PY{o}{.}\PY{n}{row\PYZus{}join}\PY{p}{(}\PY{n}{s4}\PY{p}{)}
        \PY{n}{S}
\end{Verbatim}
\texttt{\color{outcolor}Out[{\color{outcolor}6}]:}


        \begin{equation*}
        \left[\begin{matrix}-1 & -1 & 1 & 1\\1 & -1 & -1 & 1\end{matrix}\right]
        \end{equation*}



    Before defining the conectivity matrix, we begin by defining \(n\)
vectors \(e_i\) with dimension \(n\), where \(n\) is the number of nodes
in the structure, and the element \(i\) in the vector is \(1\) while the
rest are all \(0\)'s.

    \begin{Verbatim}[commandchars=\\\{\}]
{\color{incolor}In [{\color{incolor}7}]:} \PY{n}{e1} \PY{o}{=} \PY{n}{Matrix}\PY{p}{(}\PY{p}{[}\PY{l+m+mi}{1}\PY{p}{,} \PY{l+m+mi}{0}\PY{p}{,} \PY{l+m+mi}{0}\PY{p}{,} \PY{l+m+mi}{0}\PY{p}{]}\PY{p}{)}
        \PY{n}{e2} \PY{o}{=} \PY{n}{Matrix}\PY{p}{(}\PY{p}{[}\PY{l+m+mi}{0}\PY{p}{,} \PY{l+m+mi}{1}\PY{p}{,} \PY{l+m+mi}{0}\PY{p}{,} \PY{l+m+mi}{0}\PY{p}{]}\PY{p}{)}
        \PY{n}{e3} \PY{o}{=} \PY{n}{Matrix}\PY{p}{(}\PY{p}{[}\PY{l+m+mi}{0}\PY{p}{,} \PY{l+m+mi}{0}\PY{p}{,} \PY{l+m+mi}{1}\PY{p}{,} \PY{l+m+mi}{0}\PY{p}{]}\PY{p}{)}
        \PY{n}{e4} \PY{o}{=} \PY{n}{Matrix}\PY{p}{(}\PY{p}{[}\PY{l+m+mi}{0}\PY{p}{,} \PY{l+m+mi}{0}\PY{p}{,} \PY{l+m+mi}{0}\PY{p}{,} \PY{l+m+mi}{1}\PY{p}{]}\PY{p}{)}
\end{Verbatim}

    So, by making the same operations as before, but with the vectors
\(e_i\), we'll obtain the matrix \(D\) which is the transpose of the
\(C\) matrix.

    \begin{Verbatim}[commandchars=\\\{\}]
{\color{incolor}In [{\color{incolor}8}]:} \PY{n}{d1} \PY{o}{=} \PY{n}{e3} \PY{o}{\PYZhy{}} \PY{n}{e1}
        \PY{n}{d2} \PY{o}{=} \PY{n}{e4} \PY{o}{\PYZhy{}} \PY{n}{e2}
        \PY{n}{d3} \PY{o}{=} \PY{n}{e2} \PY{o}{\PYZhy{}} \PY{n}{e1}
        \PY{n}{d4} \PY{o}{=} \PY{n}{e3} \PY{o}{\PYZhy{}} \PY{n}{e2}
        \PY{n}{d5} \PY{o}{=} \PY{n}{e4} \PY{o}{\PYZhy{}} \PY{n}{e3}
        \PY{n}{d6} \PY{o}{=} \PY{n}{e1} \PY{o}{\PYZhy{}} \PY{n}{e4}

        \PY{n}{D} \PY{o}{=} \PY{p}{(}\PY{p}{(}\PY{p}{(}\PY{n}{d1}\PY{o}{.}\PY{n}{row\PYZus{}join}\PY{p}{(}\PY{n}{d2}\PY{p}{)}\PY{p}{)}\PY{o}{.}\PY{n}{row\PYZus{}join}\PY{p}{(}\PY{n}{d3}\PY{p}{)}\PY{p}{)}\PY{o}{.}\PY{n}{row\PYZus{}join}\PY{p}{(}\PY{n}{d4}\PY{p}{)}\PY{p}{)}\PY{o}{.}\PY{n}{row\PYZus{}join}\PY{p}{(}\PY{n}{d5}\PY{p}{)}\PY{o}{.}\PY{n}{row\PYZus{}join}\PY{p}{(}\PY{n}{d6}\PY{p}{)}

        \PY{n}{C} \PY{o}{=} \PY{n}{D}\PY{o}{.}\PY{n}{T}
        \PY{n}{C}
\end{Verbatim}
\texttt{\color{outcolor}Out[{\color{outcolor}8}]:}


        \begin{equation*}
        \left[\begin{matrix}-1 & 0 & 1 & 0\\0 & -1 & 0 & 1\\-1 & 1 & 0 & 0\\0 & -1 & 1 & 0\\0 & 0 & -1 & 1\\1 & 0 & 0 & -1\end{matrix}\right]
        \end{equation*}



    We'll define now a \(\Lambda\) diagonal matrix, being it's elements, the
internal compression forces of the structure.

    \begin{Verbatim}[commandchars=\\\{\}]
{\color{incolor}In [{\color{incolor}9}]:} \PY{n}{Lambda} \PY{o}{=} \PY{n}{Matrix}\PY{p}{(}\PY{p}{[}\PY{p}{[}\PY{n}{lambda1}\PY{p}{,} \PY{l+m+mi}{0}\PY{p}{]}\PY{p}{,} \PY{p}{[}\PY{l+m+mi}{0}\PY{p}{,} \PY{n}{lambda2}\PY{p}{]}\PY{p}{]}\PY{p}{)}
        \PY{n}{Lambda}
\end{Verbatim}
\texttt{\color{outcolor}Out[{\color{outcolor}9}]:}


        \begin{equation*}
        \left[\begin{matrix}\lambda_{1} & 0\\0 & \lambda_{2}\end{matrix}\right]
        \end{equation*}



    \begin{Verbatim}[commandchars=\\\{\}]
{\color{incolor}In [{\color{incolor}10}]:} \PY{o}{\PYZhy{}}\PY{n}{B}\PY{o}{*}\PY{n}{Lambda}
\end{Verbatim}
\texttt{\color{outcolor}Out[{\color{outcolor}10}]:}


        \begin{equation*}
        \left[\begin{matrix}2 \lambda_{1} & 0\\0 & 2 \lambda_{2}\end{matrix}\right]
        \end{equation*}



    And the \(\Gamma\) diagonal matrix, with it's elements the internal
stress forces.

    \begin{Verbatim}[commandchars=\\\{\}]
{\color{incolor}In [{\color{incolor}11}]:} \PY{n}{Gamma} \PY{o}{=} \PY{n}{Matrix}\PY{p}{(}\PY{p}{[}\PY{p}{[}\PY{n}{gamma1}\PY{p}{,} \PY{l+m+mi}{0}\PY{p}{,} \PY{l+m+mi}{0}\PY{p}{,} \PY{l+m+mi}{0}\PY{p}{]}\PY{p}{,} \PY{p}{[}\PY{l+m+mi}{0}\PY{p}{,} \PY{n}{gamma2}\PY{p}{,} \PY{l+m+mi}{0}\PY{p}{,} \PY{l+m+mi}{0}\PY{p}{]}\PY{p}{,} \PY{p}{[}\PY{l+m+mi}{0}\PY{p}{,} \PY{l+m+mi}{0}\PY{p}{,} \PY{n}{gamma3}\PY{p}{,} \PY{l+m+mi}{0}\PY{p}{]}\PY{p}{,} \PY{p}{[}\PY{l+m+mi}{0}\PY{p}{,} \PY{l+m+mi}{0}\PY{p}{,} \PY{l+m+mi}{0}\PY{p}{,} \PY{n}{gamma4}\PY{p}{]}\PY{p}{]}\PY{p}{)}
         \PY{n}{Gamma}
\end{Verbatim}
\texttt{\color{outcolor}Out[{\color{outcolor}11}]:}


        \begin{equation*}
        \left[\begin{matrix}\gamma_{1} & 0 & 0 & 0\\0 & \gamma_{2} & 0 & 0\\0 & 0 & \gamma_{3} & 0\\0 & 0 & 0 & \gamma_{4}\end{matrix}\right]
        \end{equation*}



    \begin{Verbatim}[commandchars=\\\{\}]
{\color{incolor}In [{\color{incolor}12}]:} \PY{n}{S}\PY{o}{*}\PY{n}{Gamma}
\end{Verbatim}
\texttt{\color{outcolor}Out[{\color{outcolor}12}]:}


        \begin{equation*}
        \left[\begin{matrix}- \gamma_{1} & - \gamma_{2} & \gamma_{3} & \gamma_{4}\\\gamma_{1} & - \gamma_{2} & - \gamma_{3} & \gamma_{4}\end{matrix}\right]
        \end{equation*}



    It's worth noting that the \(M\) matrix that contains all of the
elements of the structure, can be obtained in two ways.

\[
M = \begin{bmatrix} B & S \end{bmatrix}
\]

\[
M = N C^T
\]

    \begin{Verbatim}[commandchars=\\\{\}]
{\color{incolor}In [{\color{incolor}13}]:} \PY{n}{M} \PY{o}{=} \PY{n}{B}\PY{o}{.}\PY{n}{row\PYZus{}join}\PY{p}{(}\PY{n}{S}\PY{p}{)}
         \PY{n}{M}
\end{Verbatim}
\texttt{\color{outcolor}Out[{\color{outcolor}13}]:}


        \begin{equation*}
        \left[\begin{matrix}-2 & 0 & -1 & -1 & 1 & 1\\0 & -2 & 1 & -1 & -1 & 1\end{matrix}\right]
        \end{equation*}



    \begin{Verbatim}[commandchars=\\\{\}]
{\color{incolor}In [{\color{incolor}14}]:} \PY{n}{N}\PY{o}{*}\PY{n}{C}\PY{o}{.}\PY{n}{T}
\end{Verbatim}
\texttt{\color{outcolor}Out[{\color{outcolor}14}]:}


        \begin{equation*}
        \left[\begin{matrix}-2 & 0 & -1 & -1 & 1 & 1\\0 & -2 & 1 & -1 & -1 & 1\end{matrix}\right]
        \end{equation*}



    And due to the partition of the matrix
\(M = \begin{bmatrix} B & S \end{bmatrix}\), it simplifies the math of
the matrix \(F\):

\[
F =
\begin{bmatrix}
-B \Lambda & S \Gamma
\end{bmatrix} C
\]

    \begin{Verbatim}[commandchars=\\\{\}]
{\color{incolor}In [{\color{incolor}15}]:} \PY{n}{F} \PY{o}{=} \PY{p}{(}\PY{o}{\PYZhy{}}\PY{n}{B}\PY{o}{*}\PY{n}{Lambda}\PY{p}{)}\PY{o}{.}\PY{n}{row\PYZus{}join}\PY{p}{(}\PY{n}{S}\PY{o}{*}\PY{n}{Gamma}\PY{p}{)}\PY{o}{*}\PY{n}{C}
         \PY{n}{F}
\end{Verbatim}
\texttt{\color{outcolor}Out[{\color{outcolor}15}]:}


        \begin{equation*}
        \left[\begin{matrix}\gamma_{1} + \gamma_{4} - 2 \lambda_{1} & - \gamma_{1} + \gamma_{2} & - \gamma_{2} - \gamma_{3} + 2 \lambda_{1} & \gamma_{3} - \gamma_{4}\\- \gamma_{1} + \gamma_{4} & \gamma_{1} + \gamma_{2} - 2 \lambda_{2} & - \gamma_{2} + \gamma_{3} & - \gamma_{3} - \gamma_{4} + 2 \lambda_{2}\end{matrix}\right]
        \end{equation*}



    So now the only thing left to assume is the fact that the sum of the
internal forces in each node is \(0\). The next line of code, assumes
that every entry of the matrix is equal to \(0\), so we only have to put
tha matrix for it to yield:

    \begin{Verbatim}[commandchars=\\\{\}]
{\color{incolor}In [{\color{incolor}16}]:} \PY{n}{solve}\PY{p}{(}\PY{n}{F}\PY{p}{)}
\end{Verbatim}
\texttt{\color{outcolor}Out[{\color{outcolor}16}]:}


        \begin{equation*}
        \begin{bmatrix}\begin{Bmatrix}\gamma_{1} : \lambda_{2}, & \gamma_{2} : \lambda_{2}, & \gamma_{3} : \lambda_{2}, & \gamma_{4} : \lambda_{2}, & \lambda_{1} : \lambda_{2}\end{Bmatrix}\end{bmatrix}
        \end{equation*}



    Every force must be the same.


    % Add a bibliography block to the postdoc



    % \end{document}
