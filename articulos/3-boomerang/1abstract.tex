\section*{Resumen} 

	El boomerang es un artefacto conocido mundialmente por el interesante comportamiento que presenta al ser arojado correctamente. Su comportamiento durante el vuelo es explicado por la interacción de sus alas en movimiento con el viento, lo cual genera efectos de precesión y recostamiento. Algunas variables de las ecuaciones dinámicas necesitan ser suavizadas o promediadas para simplificar su solución por medio de un método de integración numérica. Un simulador y su interfaz gráfica fueron desarrolladas a partir de estas ecuaciones simplificadas en $Matlab_{\textregistered}$. Finalmente se llevaron a cabo experimentos con un sensor de medición inercial embarcado en un boomerang con el objetivo de recopilar datos para la futura reconstrucción de la trayectoria de vuelo.

\begingroup
\let\clearpage\relax

\section*{Abstract}


	A boomerang is an artifact known worldwide for its interesting behavior when thrown correctly. His behavior during flight can be explained by the interaction of its moving wings with the air, generating effects such as precession and lying down. Some variables of the dynamic equations need to be smoothed or averaged in order to simplify their solutions throught a numerical integration method. A simulator and its graphical interface were developed from these simplified equations in $Matlab_{\textregistered}$. Finally, some experiments were performed using an inertial measurement unit mounted on a boomerang for the future reconstruction of the flying path.

\endgroup
