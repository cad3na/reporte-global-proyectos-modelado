\section{Conclusiones}

	La solución por integración numérica de las ecuaciones de movimiento resultantes, obtenidas en el simulador, generaron trayectorias similares a las que se observan en un boomerang de retorno convencional.

	La utilización del algorimo DCM arrojó datos que muestran ser consistentes con el cambio de orientación de un boomerang durante el vuelo. Sin embargo, el giro respecto al eje Z, correspondiente al giro respecto al centro de masa del boomerang, mostró cambios demasiado pronunciados en cada periodo de muestreo (se tienen alrededor de cuatro muestras por giro), lo que significa que es necesario reducirlo algunas veces más.

	La doble integración de los datos de aceleración no arrojo información útil y será necesario revisar a detalle los puntos mencionados en la sección 4.4.2.

	La reconstrucción de la trayectoria permanece como trabajo futuro junto con la derivación de fuerzas aerodinámicas y su parametrización.
