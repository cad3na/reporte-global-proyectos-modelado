% \documentclass{article}%
% \usepackage{amsfonts}%
% \usepackage{amsmath}%
% \setcounter{MaxMatrixCols}{30}%
% \usepackage{amssymb}%
% \usepackage{graphicx}
% %TCIDATA{OutputFilter=latex2.dll}
% %TCIDATA{Version=5.50.0.2953}
% %TCIDATA{CSTFile=Math with theorems suppressed.cst}
% %TCIDATA{Created=Sunday, December 14, 2014 13:08:08}
% %TCIDATA{LastRevised=Sunday, December 14, 2014 21:32:24}
% %TCIDATA{<META NAME="GraphicsSave" CONTENT="32">}
% %TCIDATA{<META NAME="SaveForMode" CONTENT="1">}
% %TCIDATA{BibliographyScheme=Manual}
% %TCIDATA{<META NAME="DocumentShell" CONTENT="Scientific Notebook\Blank Document">}
% %TCIDATA{PageSetup=72,72,72,72,0}
% %TCIDATA{AllPages=
% %F=36,\PARA{038<p type="texpara" tag="Body Text" >\hfill \thepage}
% %}
% %BeginMSIPreambleData
% \providecommand{\U}[1]{\protect\rule{.1in}{.1in}}
% %EndMSIPreambleData
% \newtheorem{theorem}{Theorem}
% \newtheorem{acknowledgement}[theorem]{Acknowledgement}
% \newtheorem{algorithm}[theorem]{Algorithm}
% \newtheorem{axiom}[theorem]{Axiom}
% \newtheorem{case}[theorem]{Case}
% \newtheorem{claim}[theorem]{Claim}
% \newtheorem{conclusion}[theorem]{Conclusion}
% \newtheorem{condition}[theorem]{Condition}
% \newtheorem{conjecture}[theorem]{Conjecture}
% \newtheorem{corollary}[theorem]{Corollary}
% \newtheorem{criterion}[theorem]{Criterion}
% \newtheorem{definition}[theorem]{Definition}
% \newtheorem{example}[theorem]{Example}
% \newtheorem{exercise}[theorem]{Exercise}
% \newtheorem{lemma}[theorem]{Lemma}
% \newtheorem{notation}[theorem]{Notation}
% \newtheorem{problem}[theorem]{Problem}
% \newtheorem{proposition}[theorem]{Proposition}
% \newtheorem{remark}[theorem]{Remark}
% \newtheorem{solution}[theorem]{Solution}
% \newtheorem{summary}[theorem]{Summary}
% \newenvironment{proof}[1][Proof]{\noindent\textbf{#1.} }{\ \rule{0.5em}{0.5em}}
% \begin{document}

% \begin{center}
% CENTRO DE INVESTIGACIONES AVANZADAS DEL INSTITUTO POLIT\'{E}CNICO NACIONAL
%
% DEPARTAMENTO DE CONTROL AUTOM\'{A}TICO
%
% MAESTR\'{I}A EN CIENCIAS EN CONTROL AUTOM\'{A}TICO
%
% REPORTE FINAL DE PROYECTO
%
% \textbf{INTERACCI\'{O}N ENTRE EL BIENESTAR PSICOL\'{O}GICO Y EL PAISAJE
% SONORO.}
%
% ING. CONCEPCI\'{O}N JAZM\'{I}N SU\'{A}REZ POLO
%
% ING. MILCOM ELIJACK PEREGRINA OCHOA
%
% ASESOR: DR. JUAN CARLOS MART\'{I}NEZ GARC\'{I}A
%
% \bigskip
%
% DICIEMBRE 2014
%
% \bigskip
% \end{center}

% \textbf{Indice general. }
%
% Resumen \qquad2
%
% Abstract\qquad\qquad\qquad\qquad\qquad\qquad\qquad\qquad\qquad\qquad2
%
% 1. Introducci\'{o}n\qquad\qquad\qquad\qquad\qquad\qquad\qquad\qquad
% \qquad\qquad\qquad\qquad\ \ \ \ 2
%
% 2. Cronograma\qquad\qquad\qquad\qquad\qquad\qquad\qquad\qquad\ \qquad
% \qquad\qquad\qquad\ \ \ 3
%
% 3. Estado del arte\qquad\qquad\qquad\qquad\qquad\qquad\qquad\qquad\qquad
% \qquad\qquad\qquad3
%
% 3.1 Surgimiento y desarrollo del concepto de paisaje sonoro.\qquad3
%
% 3.2 Caracter\'{\i}sticas del paisaje sonoro.\qquad3
%
% 3.3 An\'{a}lisis del paisaje sonoro.\qquad4
%
% 3.4 Estructura del paisaje sonoro.\qquad5
%
% 3.5 Caracter\'{\i}sticas estructurales para el an\'{a}lisis de paisajes
% sonoros. \ \ \ 5
%
% 4. Entrevista.\qquad6
%
% 5. Parte Experimental.\qquad8
%
% 5.1 Equipo de grabaci\'{o}n.\qquad8
%
% 5.2 Grabaci\'{o}n de campo.\qquad9
%
% 5.2.1 Paisaje sonoro natural y paisaje sonoro con elemento humano.\qquad9
%
% 5.2.2 Paisaje sonoro urbano.\qquad10
%
% 5.3 Edici\'{o}n y an\'{a}lisis de las grabaciones.\qquad10
%
% 6. Dise\~{n}o de la Red Perceptr\'{o}n Multicapa con algoritmo
%
% supervisado Backpropagation tipo gradiente.\qquad12
%
% 6.1 Acondicionamiento de datos.\qquad17
%
% 6.2 Entrenamiento de la red neuronal.\qquad18
%
% 6.3 Modo de operaci\'{o}n de la aplicaci\'{o}n.\qquad23
%
% Conclusiones.\qquad24
%
% Referencias.\qquad24
%
% Bibliograf\'{\i}a consultada.\qquad24
%
% \textbf{\'{I}ndice de figuras y tablas}
%
% Figura 1. Grabadora Tascam DR40\qquad8
%
% Figura 2. Aud\'{\i}fonos Sennheiser hd 205\qquad8
%
% Figura 3. Grabadora Tascam DR40 montada en tr\'{\i}pode y con antiviento\qquad9
%
% Figura 4. Espectrograma antes y despu\'{e}s de correcciones\qquad11
%
% Figura 5. An\'{a}lisis de espectro en frecuencia paisaje natural, natural
% con\qquad12
%
% elemento humano y urbano respectivamente\qquad12
%
% Figura 6. Datos del espectrograma.\qquad12
%
% Figura 7. Representaci\'{o}n de una Neurona\qquad14
%
% Figura 8. Funci\'{o}n sigmoidea\qquad15
%
% Figura 9. Espectro analizado por Audacity\qquad18
%
% Figura 10. Organizaci\'{o}n de los datos en hojas de c\'{a}lculo\qquad18
%
% Figura 11. Gr\'{a}fica en Matlab de los datos de entrenamiento\qquad19
%
% Figura 12. Patrones de entrenamiento\qquad19
%
% Figura 13. Hipermatriz de capas de la red neuronal\qquad13
%
% Figura 14. Matriz de Bias de la red neuronal\qquad20
%
% Figura 15. Resultado para 500 \'{e}pocas\qquad21
%
% Figura 16. Resultado para 600 \'{e}pocas\qquad21
%
% Figura 17. Resultado para 800 \'{e}pocas\qquad22
%
% Figura 18. Aplicaci\'{o}n standalone\qquad23
%
% Tabla 1. Cronograma de actividades.\qquad3
%
% Tabla 2. Rangos de sonido de la OMS\qquad22

\section*{Resumen }

En el presente Proyecto, se realizar\'{o}n grabaciones de campo en tres
entornos sonoros distintos, catalogados como paisaje sonoro natural, natural
con elemento humano y urbano. A partir de las grabaciones se procedi\'{o} a un
tratamiento digital de la se\~{n}ales, los datos obtenidos se utilizaron como
referentes para el correspondiente entrenamiento de una red neuronal basada en
el modelo de Retro-propagaci\'{o}n que realiza una m\'{e}trica espacial y
temporal de la calidad de un paisaje sonoro, adem\'{a}s de obtenerse la
flexibilidad de caracterizar aquellos paisajes sonoros nocivos y su
relaci\'{o}n con el malestar psicol\'{o}gico.

\section*{Abstract }

In the current project, field recordings were performed in three different
sound environments, categorized as natural soundscape, natural soundscape with
human element and urban soundscape. From the recordings we proceeded to a
digital processing of signals, data obtained were used as references for the
corresponding back propagation neural network training, which performs a
spatial and temporal soundscape quality metric, plus we obtained the
flexibility to characterize harmful soundscapes and its relationship to
psychological distress.

\section{Introducci\'{o}n}

Contrariamente a nuestra percepci\'{o}n visual, no podemos renunciar al
sentido del o\'{\i}do, carecemos de \textquotedblleft parpados
auditivos\textquotedblright. Nuestra escucha es adem\'{a}s \textquotedblleft
omnidireccional\textquotedblright, consciente o inconscientemente, la escucha
constituye a menudo nuestro primer acercamiento y modo de comprensi\'{o}n del
entorno. Principlamente nos servimos de ella como de un \textquotedblleft
radar\textquotedblright\ que nos informa de cuanto nos rodea y que nos indica
en que hemos de fijar nuestra atenci\'{o}n, al tiempo que nos permite
descartar muchas otras fuentes de informaci\'{o}n [1].

Los sonidos que se escuchan hoy en d\'{\i}a en cualquier ciudad son muy
distintos de aquellos que pod\'{\i}an escucharse haces algunos a\~{n}os, todo
ellos debido al r\'{\i}tmico cambio mayormente acentuado a partir de la
Revoluci\'{o}n Industrial. Muchos de los sonidos naturales del propio entorno
se han visto opacados o desaparecidos del paisaje sonoro actual, la respuesta
de los habitantes de un espacio con contaminaci\'{o}n ac\'{u}stica se traduce
en malestares psicol\'{o}gicos entre otras cosas. Hoy en d\'{\i}a con el
desarrollo de la ciencia y tecnolog\'{\i}a es posible explorar, estudiar y
disfrutar los sonidos de estos tiempos, as\'{\i} como herramientas para el
an\'{a}lisis de un paisaje sonoro partiendo de su descomposici\'{o}n en
elementos. Y es justamente esta caracter\'{\i}stica la que nos permite
desarrollar herramental para describir interacciones y mejorar la calidad
ac\'{u}stica del ambiente.

\section{Cronograma.}

\begin{center}
    $%
    \begin{array}
        [c]{cc}%
        \text{\textbf{Semana}} & \text{\textbf{Actividades}}\\
        \text{1} & \text{Entrevistas}\\
        \text{2} & \text{Elecci\'{o}n de lugares y objetos agrabar}\\
        \text{3} & \text{Gesti\'{o}n de permisos de grabaci\'{o}n}\\
        \text{4} & \text{Grabaciones en campo }\\
        \text{5} & \text{Grabaciones en campo}\\
        \text{6} & \text{Edici\'{o}n de los archivos y an\'{a}lisis en el dominio de
        la frecuencia}\\
        \text{7} & \text{Extracci\'{o}n de los elementos caracter\'{\i}sticos}\\
        \text{8} & \text{Dise\~{n}o de una red neuronal artificial por algoritmo de
        retropropagaci\'{o}n}\\
        \text{9} & \text{Entrenamiento de la red neuronal}\\
        \text{10} & \text{Construcci\'{o}n de paisajes sonoros}\\
        \text{11} & \text{Construcci\'{o}n de aplicaci\'{o}n}%
    \end{array}
    $

    Tabla 1. Cronograma de actividades.
\end{center}



\section{Estado del arte}

\subsection{Surgimiento y desarrollo del concepto de paisaje sonoro.}

El t\'{e}rmino paisaje sonoro deriva de paisaje terrestre [N. del T.: en
ingl\'{e}s "soundscape" deriva de "landscape"]. El paisaje sonoro hace
referencia a cualquier ambiente ac\'{u}stico, ya sea natural, urbano, o rural,
que este formado por tres componentes: la biofon\'{\i}a: sonidos
biol\'{o}gicos no humanos que se producen en una ambiente dado, la
geofon\'{\i}a: sonidos ni humanos ni biol\'{o}gicos, como el efecto del
viento, el agua o el clima, y la antrofon\'{\i}a: el ruido que produce el ser
humano por cualquier medio [5]. Por lo tanto, el medio ambiente sonoro (o
paisaje sonoro), que es la suma de la totalidad de sonidos dentro de un
\'{a}rea definida, es un reflejo \'{\i}ntimo de -entre otros- las condiciones
sociales, pol\'{\i}ticas, tecnol\'{o}gicas y naturales del \'{a}rea. Cambios
en las mencionadas condiciones implican cambios en el medio ambiente sonoro [4].

El compositor canadiense R. Murray Schafer us\'{o} los t\'{e}rminos paisaje
sonoro ("soundscape") y ecolog\'{\i}a ac\'{u}stica para describir
cr\'{\i}ticamente nuestro medio ambiente como un campo humano-ecol\'{o}gico
ubicado entre "el sonido y el ruido". A partir de all\'{\i} desarroll\'{o} la
idea de una disciplina futurista, con claras influencias de la bauhaus$^{1}$,
el dise\~{n}o ac\'{u}stico. Su idea era juntar compositores contempor\'{a}neos
con arquitectos, dise\~{n}adores de productos e ingenieros a fin de
desarrollar sonidos para los innumerables objetos de nuestra vida cotidiana [6].

Por su parte el compositor Barry Truax diferencia lo que es un medio ambiente
s\'{o}nico de un paisaje sonoro. Para el, el primero comprende toda la
energ\'{\i}a ac\'{u}stica en un contexto dado, mientras que el segundo es la
comprensi\'{o}n de ese medio ambiente s\'{o}nico para aquellos que viven en
\'{e}l y lo est\'{a}n creando continuamente.$^{2}$ Para Truax, la audici\'{o}n
es algo fundamental, ya que constituye la interface entre el individuo y el
medio ambiente, por lo tanto el paisaje sonoro es el sistema resultante de la
suma de estos dos componentes. Para Abraham Moles el paisaje sonoro es una
secuencia corta de entre 4 y 8 segundos que incluye una idea compuesta por uno
o varios signos que nos describen algo que est\'{a} sucediendo (ideoscena).

\subsection{Caracter\'{\i}sticas del paisaje sonoro.}

\textquotedblleft La vida cotidiana tiene una banda sonora. Si no la
escuchamos, es porque ya estamos acostumbrados a o\'{\i}rla.\textquotedblright%
, nos dice el music\'{o}logo Ram\'{o}n Pelinski.

Existen una multitud de sonidos a nuestro alrededor, sonidos propios de la
naturaleza y aquellos generados cotidianamente en cada uno de nuestros
quehaceres. Sin embargo pocos son los que realmente escuchan estos sonidos,
estos paisajes que describen el lugar en el que vivimos y el entorno en el que
nos movemos [7].

De acuerdo con Schafer, las cualidades de un paisaje sonoro son:
\textquotedblleft sonidos t\'{o}nicos\textquotedblright\ (keynote sounds), un
conjunto de rasgos de identidad constituido por cuanto o\'{\i}mos de forma
distra\'{\i}da, sin atenci\'{o}n particular pues forman un continuo, un fondo
sonoro al que estamos plenamente habituados, $^{4}$ \textquotedblleft sonidos
se\~{n}ales\textquotedblright, los sonidos que existen en un primer plano y
que son escuchados de manera consciente. Son m\'{a}s que figuras de fondo y la
mayor\'{\i}a de veces representan c\'{o}digos; \textquotedblleft sonidos
importantes\textquotedblright\ (soundmarks), esto es, los sonidos que los
individuos identifican como sonidos claves de su comunidad. Murray habla
tambi\'{e}n de sonidos que se manifiestan como terreno (ground), que
interpreto como sonidos fondo (background), y de figuras que se manifiestan en
un primer plano (foreground),$^{5}$, as\'{\i} como de un tercer nivel llamado
campo (field), que es el lugar desde el cual se escucha el paisaje sonoro.

En cuanto a paisajes sonoros urbanos tomaremos la definici\'{o}n de Ricardo
Atienza, de acuerdo con \'{e}l, cada espacio urbano posee unos rasgos sonoros
caracter\'{\i}sticos que nos comunican de sus cualidades espaciales, de las
temporalidades y de los usos que lo habitan. Tales rasgos constituyen su
identidad ordinaria, cotidiana. El continuo sonoro de las ciudades no es un
\textquotedblleft ruido\textquotedblright\ neutro y arbitrario; el estudio de
sus atributos compositivos constituye un an\'{a}lisis cualitativo de las
diferentes configuraciones urbanas [1].

No podemos comprender la identidad de un lugar sin conocer primero de que modo
es habitado, recorrido y practicado un espacio. An\'{a}logamente, la identidad
de cada persona estar\'{a} vinculada en gran medida a los espacios que habite.
Esta doble interacci\'{o}n nos permite comprender la identidad de un lugar
como la expresi\'{o}n cualitativa de un espacio a trav\'{e}s de sus modos de
vida caracter\'{\i}sticos. Podriamos decir entonces que, todo fen\'{o}meno de
identidad no es sino el resultado de la tensi\'{o}n que se establece entre una
memoria sonora y una escucha futura o proyectada o bien tratarse de un proceso
din\'{a}mico tanto en las periodicidades c\'{\i}clicas de cada d\'{\i}a o de
cada estaci\'{o}n, como en la progresiva evoluci\'{o}n social y espacial de un
lugar [1].

\subsection{An\'{a}lisis del paisaje sonoro.}

Murray Schafer propone varios par\'{a}metros para el an\'{a}lisis de objetos
sonoros que forman parte de los paisajes sonoros:

1. Escuchado con claridad; claridad moderada; poca claridad; sobre el ambiente general.

2. Ocurrencia aislada; repetida; parte de un contexto m\'{a}s grande o mensaje.

3. Factores de medioambiente: sin reverberaci\'{o}n, poca reverberaci\'{o}n,
larga reverberaci\'{o}n, eco, flujo, desplazamiento.

A su vez Abraham Moles propone otros rasgos significativos del medio ambiente
sonoro, de los cuales algunos de los m\'{a}s relevantes para este proyecto son
los siguientes:

1. Numero relativo de elementos. Densidad global de los acontecimientos.

2. Complejidad del conjunto de los elementos: n\'{u}mero y variedad de las relaciones.

3. Relaci\'{o}n entre la masa de los elementos \textquotedblleft
cercanos\textquotedblright\ y la de los elementos \textquotedblleft
lejanos\textquotedblright\ (noci\'{o}n de \textquotedblleft primer
plano\textquotedblright).$^{6}$

El artista sonoro e investigador Manuel Rocha Iturbide, en su articulo
\textquotedblleft Estructura y percepci\'{o}n psicoac\'{u}stica del paisaje
sonoro electroac\'{u}stico\textquotedblright\ a\~{n}ade dos aspectos m\'{a}s
que no fueron abordados por Schafer y Moles, estos son:

1. La escucha lineal en oposici\'{o}n a la escucha no lineal de un paisaje
sonoro determinado. Por el termino lineal se refiere a una escucha en la que
no podemos concentrarnos y seguir \qquad\ \ el transcurso de los eventos, y
por no lineal, a una escucha en la que nuestra atenci\'{o}n va constantemente
de un lugar a otro, sin permitirnos asimilar o percibir continuidad.

2. El car\'{a}cter continuo o discontinuo en la estructura del paisaje sonoro.
Puede haber paisajes sonoros esencialmente continuos pero con elementos
discontinuos, o viceversa.

\subsection{Estructura del paisaje sonoro.}

Schafer define dos caracter\'{\i}sticas importantes para el an\'{a}lisis de
paisajes sonoros, los paisajes sonoros Hi-Fi y Low-Fi. En los primeros,
\textquotedblleft los sonidos se sobreponen menos frecuentemente, hay
perspectiva (amplitud de fondo)\textquotedblright\ $^{7}$. Estos paisajes se
manifiestan m\'{a}s en el campo que en la ciudad. En los segundos, Lo-Fi, los
distintos planos se empastan unos con otros, y es muy dif\'{\i}cil discernir
figuras o fondos claros. Estos paisajes son t\'{\i}picos de las grandes urbes
debidas sobre todo al ruido del tr\'{a}fico en las calles, perif\'{e}ricas y
carreteras.$^{8}$

Otros dos t\'{e}rminos que usa Schafer son gesto y textura. El primero se
refiere a una figura que constituye un \'{u}nico y distinguible evento, y el
segundo a un agregado, el efecto de manchas de imprecisi\'{o}n an\'{a}rquica y
de acciones conflictivas. En cuanto a la textura, Schafer realiza la siguiente
clasificaci\'{o}n: textura del medio ambiente escuchado: hi-fi, low-fi,
natural, humano y tecnol\'{o}gico. Por su parte, Truax habla tambi\'{e}n de la
densidad como un posible par\'{a}metro descriptivo.

\subsection{Caracter\'{\i}sticas estructurales para el an\'{a}lisis de
paisajes sonoros. }

El investigador Manuel Rocha en su mismo articulo identifica cuatro tipos de
paisajes sonoros, para este proyecto de investigaci\'{o}n unicamente se
utilizo \textit{paisajes sonoros naturales}, \textit{paisajes sonoros
naturales con elemento humano} y \textit{paisajes sonoros urbano}s para
clasificar los entornos sonoros de estudio.

El paisaje sonoro natural corresponde a un entorno donde principalmente se
detecta sonidos del agua, del viento, sinfon\'{\i}as pajariles, sonidos de
insectos, etc. El paisaje sonoro natural con elemento humano, el cual se
utilizo en el proyecto para clasificar espacios naturales en los que
adem\'{a}s de los sonidos propios de un paisaje natural podemos encontrarnos
con sonidos de caracter discontinuo propios de la intervenci\'{o}n del humano.
\ El paisaje sonoro urbanos presenta una mayor riqueza de fuentes sonoras
respecto a los anteriores pues en el podemos encontrar diferentes sonidos, ya
sean humanos (voces, pasos, etc.), mec\'{a}nicos (tr\'{a}fico vehicular,
m\'{a}quinas, etc.) en incluso sonidos naturales (p\'{a}jaros, fuentes de
agua, etc.), los cuales contribuyen a la diferenciaci\'{o}n de m\'{u}ltiples
paisajes sonoros urbanos, los cuales coexisten y, en m\'{u}ltiples casos, se
combinan dentro la aglomeraci\'{o}n urbana.

Importante mencionar que de acuerdo con el investigador \ Manuel Rocha, el
tiempo m\'{\i}nimo para poder comprender la estructura de un paisaje sonoro,
as\'{\i} como para analizar su complejidad, es de unos 40 segundos. La
decisi\'{o}n respecto a caminar y como girar el campo est\'{e}reo de los
micr\'{o}fonos durante el proceso de grabaci\'{o}n, son decisiones de
car\'{a}cter estructural y de composici\'{o}n activa consciente.

\section{Entrevista.}

La presente entrevista se realiz\'{o} al experto en paisaje sonoro,
grabaci\'{o}n de campo y experimentaci\'{o}n sonora Enrique Maraver Aguirre.

\textbf{1. Comentenos un poco sobre usted. }

- \textit{Mi nombre es Enrique Maraver Aguirre, nacido en el Distrito Federal.
Egresado del Instituto Politecnico Nacional como Ingeniero Quimico Industrial,
posteriormente me involucre en la cuesti\'{o}n del sonido hace casi 10
a\~{n}os. Empec\'{e} la produccion de trabajos artisticos hace aproximadamente
7 a\~{n}os he publicado en distintos sellos en Portugal, Londres, Espa\~{n}a,
Colombia, M\'{e}xico, etc. He dado algunos talleres sobre paisajes sonoros y
ecologia acustica en Espa\~{n}a y en M\'{e}xico. Entre mis proyectos se
encuentran las producciones en grabaci\'{o}n de campo, algunos proyectos de
experimentaci\'{o}n sonora partiendo de la naturalidad del sonido como una
fuente de creaci\'{o}n m\'{a}s all\'{a} artistica sobre todo como un medio de
sensibilizaci\'{o}n de la escucha y de incitar a la gente de que nuestro
sentido auditivo es igual de importante que los dem\'{a}s y que esto conlleva
a una cuesti\'{o}n de sensibilidad y de conciencia.-}

\textbf{2. Comentenos sobre su experiencia en la grabaci\'{o}n de paisajes
sonoros.}

- \textit{He trabajado con 3 equipos para la grabaci\'{o}n de campo que es
Tascam, Zoom y Roland. Actualmente trabajo con Zoom y Tascam, a menudo he
trabajado m\'{a}s con Tascam; la DR40 es muy vers\'{a}til, c\'{o}moda y
pr\'{a}ctica, tiene la modalidad de grabar en mono, calidad dual y cuatro
canales.-}

\textbf{a) \textquestiondown Qu\'{e} t\'{e}cnicas de grabaci\'{o}n utiliza con
este equipo? }

\textit{- Con la cuesti\'{o}n interna del grabador es en modalidad d\'{u}o o
cuatro canales, la t\'{e}cnica m\'{a}s bien la uso en el entorno como la
direcci\'{o}n del equipo, la modulaci\'{o}n de la entrada del audio, una
t\'{e}cnica especifica con el grabador no hay me gusta explorar a veces lo que
el mismo grabador interactuar con distintos niveles de decibeles, tipos de
direcciones generalmente con la configuraci\'{o}n X-Y, la A-B es m\'{a}s
ruidosa.-}

\textbf{b) \textquestiondown Radio m\'{a}ximo de detecci\'{o}n de sonidos?}

\textit{-Depende de la entrada de audio, trae una modulaci\'{o}n de entrada de
audio que va desde 100 hasta 0. Depende del entorno en el que te encuentres,
por ejemplo si quieres capturar una fuente que est\'{e}s a menos de un metro o
dos vas modulando la entrada de audio a un porcentaje de 30\% aproximadamente
si quieres capturar un sonido en un radio de 20m modulas a una cuesti\'{o}n de
50\%-60\% . Yo no lo uso a m\'{a}s del 50\% porque la calidad de la
grabaci\'{o}n se hace m\'{a}s \'{a}spera empieza a capturar todo lo que pase y
se saturan los micr\'{o}fonos. Lo que hace dif\'{\i}cil percibir y editar al
mismo tiempo. Es muy importante realizar pruebas de escucha independientemente
de la modulaci\'{o}n y configuraci\'{o}n deseada, esto es experimentar esta
cuesti\'{o}n de espacio si es necesario desplazarse para capturar los sonidos
que se requieran.-}

\textbf{c. \textquestiondown C\'{o}mo realiza la selecci\'{o}n de puntos
ac\'{u}sticos? }

\textit{- Hay una teor\'{\i}a en la cual est\'{a} plasmada por Murray Schafer,
Barry Truax, la escuela actual de Chris Watson en Londres. Cada uno en esta
cuesti\'{o}n es muy amplio desde un entorno natural, urbano, esto va
dependiendo de lo que uno quiera analizar. Las iniciativas que uso son
normalmente son m\'{a}s propias.-- }

\textbf{3. \textquestiondown Cu\'{a}les son las consideraciones t\'{e}cnicas a
la hora de realizar las grabaciones de audio? }

\qquad\textit{- Normalmente la mejor calidad de grabaci\'{o}n es en formato
WAV, el Tascam DR40 trae la modalidad de grabar en MP3 y WAV. Para m\'{\i} el
mejor registro que puedes hacer con este grabador es en formato WAV a 48.1
kHz. Esto se recomienda como m\'{\i}nimo en calidad, normalmente realizo las
grabaciones arriba de 24 bits.- }

\textit{- Del montaje de la grabadora depende mucho un paisaje sonoro bien o
mal capturado o m\'{a}s all\'{a} de un paisaje sonoro una buena o mala
grabaci\'{o}n. Evidentemente hay que tener un soporte para evitar deteriorar
una continuidad en el paisaje tan solo con tocar la misma grabadora, lo cual
conlleva editarlo lo cual no indica que pierda la naturalidad pero te conlleva
a hacer ajustes que pueden causarte inconvenientes. Utilizar tambi\'{e}n un
cubre polvos que ayuda a que no se da\~{n}en los micros internos del grabador
as\'{\i} como protegerlos del viento para evitar que la entrada de audio se
sature de informaci\'{o}n.-}

\qquad\textit{- Para investigaci\'{o}n sugiero grabaciones tomas en distintos
tiempos, en cuesti\'{o}n de tiempo no hay una exactitud depende del punto que
uno quiera capturar.-}

\textbf{4. Comentenos sobre la estructuraci\'{o}n del paisaje sonoro.}

\textit{- Trato de editar lo menos posible mis grabaciones, lo \'{u}nico que
generalmente hago es bajar o subir decibeles, cortar o poner entradas
efectuarle m\'{a}s tratamientos seria afectar la cuesti\'{o}n sonora natural
de lo capturado por lo que evito esta cuesti\'{o}n. El ajuste generalmente se
realiza durante la grabaci\'{o}n regulando la entrada de audio posteriormente
en la edici\'{o}n modulo la cuesti\'{o}n del volumen y los dB con lo cual se
elimina un poco el ruido, coloco un filtro de reducci\'{o}n de ruido
normalmente del 2\%. }Para la edici\'{o}n sugiero un soporte con Adobe
Audition, Audacity, Pro Tools, etc. \ El enfoque de espacio es muy importante
en el paisaje sonoro, porque determina la informaci\'{o}n ac\'{u}stica que hay
en el sitio, la geolocalizaci\'{o}n del lugar da mucha informaci\'{o}n de la
diversidad sonora y cultural. El paisaje sonoro es la manifestaci\'{o}n
ac\'{u}stica del entorno pero tambi\'{e}n el entorno lo hacemos nosotros, la
diversidad en la cual nos encontramos y en la cual nos desarrollamos es muy
importante es esta espacialidad del lugar.-

\textbf{5. \textquestiondown Qu\'{e} aplicaciones has dado a los paisajes
sonoros?}

\textit{- Toda la gama de informaci\'{o}n que te puede representar el sonido
es muy importante m\'{a}s all\'{a} de lo que estas escuchando, m\'{a}s
all\'{a} de lo que uno est\'{a} interpretando porque puede mantener la riqueza
de un ambiente cuidado, las aplicaciones que he dado en mi trabajo han sido de
sensibilizar al o\'{\i}do y representar una cuesti\'{o}n cultural, porque
muestro una parte de la sonoridad de una regi\'{o}n a otros o\'{\i}dos. -}

\section{Parte Experimental.}

\subsection{Equipo de grabaci\'{o}n.}

a) Grabadora port\'{a}til con microfonos integrados.

En el transcurso del proyecto se utilizo una grabadora port\'{a}til que
incorpora microfonos en el propio cuerpo de la grabadora, pudiendo utilizarlas
sin necesidad de tener que adaptarles microfonos externos. La caracteristica
com\'{u}n de estas grabadoras es que sus dimensiones son reducidas por lo que
su manipulaci\'{o}n en grabaciones de campo es sencilla. Las grabaciones
obtenidas en el trasncurso del proyecto se realizar\'{o}n con la grabadora
port\'{a}til Tascam DR40, la cual puede grabar minimamente en calidad
cinematogr\'{a}fica (48 kHz, 24 bits).

\begin{center}%
%TCIMACRO{\FRAME{dtbphF}{2.674in}{1.4399in}{0pt}{}{}{Figure}%
%{\special{ language "Scientific Word";  type "GRAPHIC";
%maintain-aspect-ratio TRUE;  display "USEDEF";  valid_file "T";
%width 2.674in;  height 1.4399in;  depth 0pt;  original-width 7.8231in;
%original-height 4.1978in;  cropleft "0";  croptop "1";  cropright "1";
%cropbottom "0";  tempfilename 'NGLTU000.wmf';tempfile-properties "XPR";}}}%
%BeginExpansion
\begin{center}
\includegraphics[
natheight=4.197800in,
natwidth=7.823100in,
height=1.4399in,
width=2.674in
]%
{imagenes/5-acustico/Figura1.jpg}%
\end{center}
%EndExpansion


Figura 1. Grabadora Tascam DR40
\end{center}

b) Audifonos sennheiser hd 205.

Para realizar grabaciones de campo se utilizan audifonos de tipo cerrado para
aislar lo m\'{a}ximo posible el ruido externo y tener una mayor precisi\'{o}n
en la escucha de la grabaci\'{o}n. Los audifonos seleccionados para tal fin
fue el Sennheiser hd 205 pricipalmente por su caracter\'{\i}stica de respuesta
en frecuencia,18-20 kHz.%

%TCIMACRO{\FRAME{dtbphF}{2.4855in}{1.6016in}{0pt}{}{}{Figure}%
%{\special{ language "Scientific Word";  type "GRAPHIC";
%maintain-aspect-ratio TRUE;  display "USEDEF";  valid_file "T";
%width 2.4855in;  height 1.6016in;  depth 0pt;  original-width 5.6559in;
%original-height 3.6357in;  cropleft "0";  croptop "1";  cropright "1";
%cropbottom "0";  tempfilename 'NGLTU001.wmf';tempfile-properties "XPR";}}}%
%BeginExpansion
\begin{center}
\includegraphics[
natheight=3.635700in,
natwidth=5.655900in,
height=1.6016in,
width=2.4855in
]%
{imagenes/5-acustico/Figura2.jpg}%
\end{center}
%EndExpansion


\begin{center}
Figura 2. Audifonos Sennheiser hd 205
\end{center}

c) Accesorios:

\qquad i) Tr\'{\i}pode: El tr\'{\i}pode es un soporte fijo para el microfono.
Lo utilizamos principalmente para no causar ningun tipo de ruido debido al
manejo, ni al movimiento del grabador.

\qquad ii) Antiviento. El antiviento es una cubierta para la grabadora digital
port\'{a}til dise\~{n}ada para mitigar las frecuencias asociadas al viento. Se
podria decir que es el accesorio m\'{a}s importante para realizar grabaciones
de campo.%

%TCIMACRO{\FRAME{dtbphF}{1.6786in}{2.1862in}{0pt}{}{}{Figure}%
%{\special{ language "Scientific Word";  type "GRAPHIC";
%maintain-aspect-ratio TRUE;  display "USEDEF";  valid_file "T";
%width 1.6786in;  height 2.1862in;  depth 0pt;  original-width 4.427in;
%original-height 5.7813in;  cropleft "0";  croptop "1";  cropright "1";
%cropbottom "0";  tempfilename 'NGLTU002.wmf';tempfile-properties "XPR";}}}%
%BeginExpansion
\begin{center}
\includegraphics[
natheight=5.781300in,
natwidth=4.427000in,
height=2.1862in,
width=1.6786in
]%
{imagenes/5-acustico/Figura3.jpg}%
\end{center}
%EndExpansion


\begin{center}
Figura 3. Grabadora Tascam DR40 montada en tr\'{\i}pode y con antiviento
\end{center}

\subsection{Grabaci\'{o}n de campo.}

La primera caracter\'{\i}stica del paisaje sonoro corresponde al formato en el
que se grab\'{o} y los micr\'{o}fonos utilizados, Durante las grabaciones en
campo se utiliz\'{o} la grabadora de mano Tascam DR40 configurada para un
formato de grabaci\'{o}n est\'{e}reo y una t\'{e}cnica microf\'{o}nica X-Y,
los dos micr\'{o}fonos internos de est\'{a} grabadora son tipo condensador.

Se realiz\'{o} grabaciones de paisajes sonoros en tres contextos diferentes,
en cada uno de ellos la t\'{e}cnica de grabaci\'{o}n es muy similar
dependiendo del modo y forma en que se dese\'{o} capturar los instantes
sonoros, eventualmente nos desplaz\'{a}bamos un poco para detectar con mayor
nitidez algunos de los elementos m\'{a}s relevantes del paisaje. A
continuaci\'{o}n se anexa la metodolog\'{\i}a que se utiliz\'{o} para cada caso.

\subsubsection{Paisaje sonoro natural, natural con elemento humano
y urbano.}

En los inicios del proyecto se busc\'{o} un \'{a}rea geogr\'{a}fica dentro del
Distrito Federal que pudiera proveernos de un paisaje sonoro con las
caracter\'{\i}sticas propias de un entorno natural, la elecci\'{o}n del
espacio de estudio se seleccion\'{o} ajust\'{a}ndose en torno a los recursos
materiales y econ\'{o}micos con los que se dispon\'{\i}a. As\'{\i} el \'{a}rea
dispuesta para este fin fue la Reserva Ecol\'{o}gica del Pedregal de San
\'{A}ngel de la UNAM (REPSA), de la cual se seleccionaron dos \'{a}reas: paseo
de las esculturas y centro escult\'{o}rico. Para el acceso a estas \'{a}reas
se requiere tramitar el permiso correspondiente ante la SEREPSA (Secretaria de
la Reserva Ecol\'{o}gica del Pedregal de San \'{A}ngel), el cual se recomienda
realizar con al menos dos semanas de anticipaci\'{o}n, una vez liberados los
permisos se procedi\'{o} a la identificaci\'{o}n y localizaci\'{o}n de puntos ac\'{u}sticos.

Las primeras grabaciones se realizaron en formato est\'{e}reo, a 44.1 kHz y 16
bits, con la t\'{e}cnica microf\'{o}nica A-B, monitoreando en todo momento la
grabaci\'{o}n con aud\'{\i}fonos. Tras realizar los primeros an\'{a}lisis de
las grabaciones se not\'{o} la inevitable intervenci\'{o}n de los sonidos
propios de la urbe (claxon, motor de autom\'{o}viles, martilleos, etc.), pocos
eran los puntos ac\'{u}sticos en los que no se detectara algo similar y es por
esto que se decidi\'{o} clasificar a las grabaciones obtenidas en Paseo de las
Esculturas como paisajes naturales con el elemento humano.

Despu\'{e}s de las primeras grabaciones tuvo lugar la primera entrevista con
el experto en paisaje sonoro, grabaci\'{o}n de campo y experimentaci\'{o}n
sonora Enrique Maraver Aguirre. En la cual se tuvo la oportunidad de
identificar los factores que deb\'{\i}an corregirse y algunos m\'{a}s que
deb\'{\i}an ser a\~{n}adidos, entre estos:

\textbullet\ Utilizar antiviento en la grabadora port\'{a}til Tascam DR40

\textbullet\ Grabaci\'{o}n con calidad m\'{\i}nima cinematogr\'{a}fica a 24
bits y 48 kHz

\bigskip\textbullet\ Ajuste del nivel de entrada entre 38\%-50\% para evitar
que los sonidos grabados distorsionen debido a se\~{n}ales de entrada
demasiado potentes o que sean inaudibles por entradas demasiado d\'{e}biles en
comparaci\'{o}n con el ruido de fondo. As\'{\i} como tambi\'{e}n una
reducci\'{o}n de picos a -12 dB.

\qquad\textbullet\qquad T\'{e}cnica microf\'{o}nica X-Y

\qquad\textbullet\qquad Monitoreo con aud\'{\i}fonos, marcas recomendadas
SENNHEISER o BOSE.

Las subsecuentes grabaciones que se obtuvieron se realizaron en base a tales
recomendaciones. Cabe comentar tambien que se tuvieron algunas inconvenientes
en cuanto al acceso a las instalaciones de la REPSA debidas a las festividades
del 2 de noviembre y despu\'{e}s a las marchas realizadas por estudiantes de
la UNAM, aun cuando se contaba con los permisos.

Por otra parte se eligi\'{o} para las grabaciones del paisaje sonoro natural
el Parque Nacional \textquotedblleft La Malinche\textquotedblright, localizado
en el estado Tlaxcala. Las grabaciones se llevaron a cabo en un horario entre
9:00 a.m. y 12:00 p.m. a lo largo de la misma ruta durante las dos sesiones de
grabaci\'{o}n a un nivel de entrada entre 40\% y 50\%, con el correspondiente
formato de grabaci\'{o}n antes mencionado y el equipo complementario.

Para la captura de un paisaje sonoro urbano se seleccion\'{o} El Mexipuerto
Ciudad Azteca, las grabaciones se realizar\'{o}n con las caracter\'{\i}sticas
t\'{e}cnicas utilizadas en los dos casos anteriores.

\subsection{Edici\'{o}n y an\'{a}lisis de las grabaciones.}

Para la edici\'{o}n y el an\'{a}lisis de las grabaciones se procedi\'{o} de la
siguiente manera:

\qquad a) Se realiz\'{o} una base de datos de las grabaciones obtenidas, los
datos capturados corresponden a la ubicaci\'{o}n geogr\'{a}fica (lecturas con
GPS durante las grabaciones en paisaje sonoro natural), tiempo de
grabaci\'{o}n, d\'{\i}a y fecha.

\qquad b) Posteriormente mediante el programa de c\'{o}mputo de c\'{o}digo
abierto Audacity se realizaron correcciones en los audios para eliminar los
sonidos capturados propios de la manipulaci\'{o}n del equipo durante las
grabaciones, la herramienta b\'{a}sica aparte de la escucha atenta es
visualizar el espectrograma del audio como se muestra en la siguiente figura.

\begin{center}%
%TCIMACRO{\FRAME{dtbphF}{3.2906in}{2.303in}{0pt}{}{}{Figure}%
%{\special{ language "Scientific Word";  type "GRAPHIC";
%maintain-aspect-ratio TRUE;  display "USEDEF";  valid_file "T";
%width 3.2906in;  height 2.303in;  depth 0pt;  original-width 12.5519in;
%original-height 8.7605in;  cropleft "0";  croptop "1";  cropright "1";
%cropbottom "0";  tempfilename 'NGLTU003.wmf';tempfile-properties "XPR";}}}%
%BeginExpansion
\begin{center}
\includegraphics[
natheight=8.760500in,
natwidth=12.551900in,
height=2.303in,
width=3.2906in
]%
{imagenes/5-acustico/Figura4.jpg}%
\end{center}
%EndExpansion


Figura 4. Espectrograma antes y despu\'{e}s de correcciones
\end{center}

\qquad c) Se realiz\'{o} la extracci\'{o}n de sonidos caracter\'{\i}sticos
para la posterior construcci\'{o}n de los paisajes sonoros, utilizando la
herramienta de espectrograma en Audacity.

\qquad d) Por otra parte, las grabaciones se fraccionaron en audios con un
minuto de duraci\'{o}n para homogenizar el tratamiento de la informaci\'{o}n.
Despu\'{e}s, cada uno de estos se edit\'{o} en Audacity para el an\'{a}lisis
de su espectro en el dominio de la frecuencia, con las siguientes propiedades:

\qquad\qquad i) Algoritmo: Espectro

\qquad\qquad ii) Funci\'{o}n: Ventana de Hanning

\qquad\qquad iii) Tama\~{n}o de muestreo: 1024 bits

\qquad\qquad iv) Eje: Frecuencia lineal

\begin{center}%
%TCIMACRO{\FRAME{dtbphF}{5.1232in}{1.7841in}{0pt}{}{}{Figure}%
%{\special{ language "Scientific Word";  type "GRAPHIC";
%maintain-aspect-ratio TRUE;  display "USEDEF";  valid_file "T";
%width 5.1232in;  height 1.7841in;  depth 0pt;  original-width 12.8433in;
%original-height 4.4477in;  cropleft "0";  croptop "1";  cropright "1";
%cropbottom "0";  tempfilename 'NGLTU004.wmf';tempfile-properties "XPR";}}}%
%BeginExpansion
\begin{center}
\includegraphics[
natheight=4.447700in,
natwidth=12.843300in,
height=1.7841in,
width=5.1232in
]%
{imagenes/5-acustico/Figura5a.jpg}%
\end{center}
%EndExpansion
%

%TCIMACRO{\FRAME{dtbphF}{5.1923in}{1.8066in}{0pt}{}{}{Figure}%
%{\special{ language "Scientific Word";  type "GRAPHIC";
%maintain-aspect-ratio TRUE;  display "USEDEF";  valid_file "T";
%width 5.1923in;  height 1.8066in;  depth 0pt;  original-width 12.8961in;
%original-height 4.4581in;  cropleft "0";  croptop "1";  cropright "1";
%cropbottom "0";  tempfilename 'NGLTU005.wmf';tempfile-properties "XPR";}}}%
%BeginExpansion
\begin{center}
\includegraphics[
natheight=4.458100in,
natwidth=12.896100in,
height=1.8066in,
width=5.1923in
]%
{imagenes/5-acustico/Figura5b.jpg}%
\end{center}
%EndExpansion
%

%TCIMACRO{\FRAME{dtbphF}{5.1716in}{1.7884in}{0pt}{}{}{Figure}%
%{\special{ language "Scientific Word";  type "GRAPHIC";
%maintain-aspect-ratio TRUE;  display "USEDEF";  valid_file "T";
%width 5.1716in;  height 1.7884in;  depth 0pt;  original-width 12.8226in;
%original-height 4.4062in;  cropleft "0";  croptop "1";  cropright "1";
%cropbottom "0";  tempfilename 'NGLTU006.wmf';tempfile-properties "XPR";}}}%
%BeginExpansion
\begin{center}
\includegraphics[
natheight=4.406200in,
natwidth=12.822600in,
height=1.7884in,
width=5.1716in
]%
{imagenes/5-acustico/Figura5c.jpg}%
\end{center}
%EndExpansion


Figura 5. An\'{a}lisis de espectro en frecuencia paisaje natural, natural con
elemento humano y urbano respectivamente

\bigskip
\end{center}

Los datos obtenidos se exportan a un archivo .tex (vease figura 6). El tipo de
unidad exportado es el dB (FS) (decibel Full Scale), es una raz\'{o}n
logar\'{\i}tmica entre la intensidad grabada y una de referencia. Esta
intensidad de referencia est\'{a} ligada al equipo que se utiliz\'{o} para la
grabaci\'{o}n y en el hardware del computador y es de 0 dB.%

%TCIMACRO{\FRAME{dtbphF}{1.5791in}{2.0617in}{0pt}{}{}{Figure}%
%{\special{ language "Scientific Word";  type "GRAPHIC";
%maintain-aspect-ratio TRUE;  display "USEDEF";  valid_file "T";
%width 1.5791in;  height 2.0617in;  depth 0pt;  original-width 4.99in;
%original-height 6.5311in;  cropleft "0";  croptop "1";  cropright "1";
%cropbottom "0";  tempfilename 'NGLTU007.wmf';tempfile-properties "XPR";}}}%
%BeginExpansion
\begin{center}
\includegraphics[
natheight=6.531100in,
natwidth=4.990000in,
height=2.0617in,
width=1.5791in
]%
{imagenes/5-acustico/Figura6.jpg}%
\end{center}
%EndExpansion


\begin{center}
Figura 6. Datos del espectrograma.
\end{center}

e) En tablas de Excel se registr\'{o} la matriz de datos correspondientes al
espectro de cada audio, de \'{e}stos \'{u}nicamente se exporto a Matlab los
vectores correspondientes a la intensidad del sonido (dB), se aplic\'{o}
normalizaci\'{o}n y posteriormente cada nuevo vector se exporto a Excel para
su subsecuente uso en el entrenamiento de la red neuronal artificial.

\section{Dise\~{n}o de la Red Perceptr\'{o}n Multicapa con algoritmo supervisado Backpropagation tipo gradiente.}

\subsection{Principios b\'{a}sicos.}

Las observaciones de la naturaleza y la de nosotros mismos son muy
inspiradoras a la hora de crear t\'{e}cnicas y algoritmos aplicables en
inteligencia Artificial. Son muchos autores que hacen comparaciones entre los
procesadores que hacen funcionar nuestros ordenadores y el cerebro, De hecho
llegan a hacer comparaciones entre el n\'{u}mero transistores y el n\'{u}mero
de neuronas de nuestro cerebro, No hace muchos a\~{n}os aventuraban que cuando
el n\'{u}mero de transistores igualaran al n\'{u}mero de neuronas del cerebro,
podr\'{\i}amos crear ordenadores inteligentes como nosotros mismos.

Lo cierto es que no estamos tan lejos de construir procesadores con un
n\'{u}mero similar de transistores al n\'{u}mero de neuronas del cerebro de
algunos animales; sin embargo estamos muy lejos de poder imitar su cerebro y
la complejidad de sus capacidades. De hecho, la velocidad a la que se activan
los transistores de los procesadores actuales, es muy superior a la velocidad
de activaci\'{o}n de las neuronas humanas.

La respuesta de \textquestiondown por qu\'{e} estamos muy lejos de emular
dichas capacidades?, se halla en el diferente modelo de procesamiento en que
opera un procesador, esto es, el procesador de una computadora opera de manera
secuencial, mientras que el cerebro opera de manera paralela y concurrente, es
por eso que el campo de procesamiento en paralelo es una de los campos de
investigaci\'{o}n donde se invierten m\'{a}s esfuerzos actualmente.

Las Redes Neuronales artificiales o RNAs son un intento de emular la forma de
trabajar del cerebro humano, y aunque estamos lejos de alcanzar su misma
capacidad son un instrumento de gran potencia para la gran cantidad de aplicaciones.

Una neurona animal est\'{a} compuesta por un n\'{u}cleo rodeada de millones de
conexiones que la unen a otras neuronas. Estas conexiones se denominan sin\'{a}psis.

Las conexiones se realizan mediante dos tipos de neurotransmisores, las
dendritas y los axones. Seg\'{u}n la Neurociencia actual parece que una
neurona funciona de manera similar a un transistor, es decir en un momento
dado esta puede estar activa o no activa. En realidad no es as\'{\i}
exactamente, pero podemos decir que se acerca mucho a este modelo

Las neuronas se activan en funci\'{o}n de las dendritas. Las dendritas
transportan se\~{n}ales el\'{e}ctricas desde otras neuronas. Cuando la
cantidad de dendritas alcanzan un umbral determinado, la neurona se activa y
env\'{\i}a se\~{n}ales el\'{e}ctricas a otras neuronas a trav\'{e}s de los axones.

\begin{center}
\qquad\qquad\qquad\qquad%
%TCIMACRO{\FRAME{itbphF}{2.7155in}{2.0141in}{0in}{}{}{1.png}%
%{\special{ language "Scientific Word";  type "GRAPHIC";
%maintain-aspect-ratio TRUE;  display "USEDEF";  valid_file "F";
%width 2.7155in;  height 2.0141in;  depth 0in;  original-width 7.587in;
%original-height 5.6135in;  cropleft "0";  croptop "1";  cropright "1";
%cropbottom "0";
%filename '../REPORTE PARTE 1/Milcom/1.png';file-properties "NPEU";}}}%
%BeginExpansion
{\includegraphics[
natheight=5.613500in,
natwidth=7.587000in,
height=2.0141in,
width=2.7155in
]%
{imagenes/5-acustico/1.png}%
}%
%EndExpansion


\qquad\qquad\qquad\qquad Figura 7. Representaci\'{o}n de una Neurona
\end{center}

El Perceptr\'{o}n es un modelo simple de una neurona que permite presentar los
conceptos b\'{a}sicos de c\'{o}mo opera la red aqu\'{\i} dise\~{n}ada. Al
igual que una neurona real, al Perceptron llegan se\~{n}ales de entrada y
saldr\'{a} una se\~{n}al que ser\'{a} la salida de una funci\'{o}n de
activaci\'{o}n. Adem\'{a}s a cada una de estas entradas se le asigna una valor
llamado peso w (por la palabra weight en Ingl\'{e}s) que da un significado de
la fuerza de conexi\'{o}n entre la se\~{n}al de entrada y la neurona,
tambi\'{e}n, por cada neurona existe un valor del umbral de disparo (b) que
est\'{a} entre 0 y 1.

A esta configuraci\'{o}n se le conoce como; red neuronal simple debido a que
est\'{a} compuesta de una capa, es decir un solo bloque de neuronas entre las
entradas y la salida de la red.

La tarea del algoritmo es ajustar los pesos w y el valor de b a trav\'{e}s de
un proceso llamado entrenamiento. A este tipo de redes se les llama
aprendizaje supervisado ya que durante el entrenamiento se va proveyendo
ejemplos a la red y seg\'{u}n la respuesta de la red comparada con la
respuesta esperada, se ajustan los valores correspondientes. Dicho de otra
manera para cada ejemplo habr\'{a} que indicar a la red neuronal cu\'{a}l es
el resultado que deber\'{\i}a darse a la salida.

El uso de redes Perceptr\'{o}n simples est\'{a} restringido a resolver
problemas linealmente separables, ya sea en $%
%TCIMACRO{\U{211d} }%
%BeginExpansion
\mathbb{R}
%EndExpansion
^{2}$ o en $%
%TCIMACRO{\U{211d} }%
%BeginExpansion
\mathbb{R}
%EndExpansion
^{3}$, pero debido la forma gr\'{a}fica de los datos obtenidos, necesitamos
m\'{a}s capas en la red.

Sin entrar en detalles, se puede decir que si se organiza un conjunto de
neuronas formando una red, se consiguen diferentes niveles de complejidad en
las particiones del plano o de cualquier otra dimensi\'{o}n. A la red de
retro-propagaci\'{o}n es una red neuronal artificial que tienen la
particularidad de que cada neurona est\'{a} conectada a todas las neuronas de
la capa anterior, a las neuronas de la primera capa se les llama neuronas de
capa de entrada, al conjunto de neuronas de la \'{u}ltima capa se les llama
capa de salida y las neuronas que est\'{a}n en medio se les llama neuronas
escondidas u ocultas [10].

Con redes de una capa podemos resolver problemas linealmente separables, con
redes de una capa oculta se pueden resolver problemas que son separables
mediante curvas y con redes de dos capas ocultas se pueden resolver problemas
en los que se den separaciones arbitrarias, por lo que en principio, no tiene
mucho sentido usar m\'{a}s de dos capas ocultas en la red, ya que como se ve
m\'{a}s adelante se trabajar\'{a} con el espetro de se\~{n}ales de audio que
son separables mediante curvas suaves.

La funci\'{o}n de activaci\'{o}n que utilizan las redes perceptron multicapa
debe de ser una funci\'{o}n derivable, esta funci\'{o}n es la funci\'{o}n
sigmoidea descrita a continuaci\'{o}n.$\qquad\qquad\qquad\qquad\qquad\qquad$

\begin{center}
$\qquad\qquad\qquad\qquad\qquad\varphi(x)=\frac{1}{1+e^{-x}}$

\qquad\qquad%
%TCIMACRO{\FRAME{itbpFX}{2.9948in}{1.9969in}{0in}{}{}{Plot}%
%{\special{ language "Scientific Word";  type "MAPLEPLOT";  width 2.9948in;
%height 1.9969in;  depth 0in;  display "USEDEF";  plot_snapshots TRUE;
%mustRecompute FALSE;  lastEngine "MuPAD";  xmin "-5";  xmax "5";
%xviewmin "-5.0010000010002";  xviewmax "5.0010000010002";
%yviewmin "5.55111512312578E-17";  yviewmax "1";  plottype 4;
%axesFont "Times New Roman,12,0000000000,useDefault,normal";  numpoints 100;
%plotstyle "patch";  axesstyle "normal";  axestips FALSE;  xis \TEXUX{x};
%var1name \TEXUX{$x$};  function \TEXUX{$\frac{1}{1+e^{-x}}$};
%linecolor "black";  linestyle 1;  pointstyle "point";  linethickness 1;
%lineAttributes "Solid";  var1range "-5,5";  num-x-gridlines 100;
%curveColor "[flat::RGB:0000000000]";  curveStyle "Line";
%VCamFile 'NGLTU000.xvz';valid_file "T";
%tempfilename 'NGLTU008.wmf';tempfile-properties "XPR";}}}%
%BeginExpansion
{\fbox{\includegraphics[
natheight=1.996900in,
natwidth=2.994800in,
height=1.9969in,
width=2.9948in
]%
{imagenes/5-acustico/Final_portable__1.png}%
}}%
%EndExpansion


Figura 8. Funci\'{o}n sigmoide
\end{center}

Se pueden utilizar otro tipo de funciones como la tangente hiperb\'{o}lica
pero en este caso sigue la sugerencia de [10] para usar la funci\'{o}n sigmoidea.

\subsection{Fases del algoritmo de retro-propagaci\'{o}n.}

El algoritmo de retro-propagaci\'{o}n tiene tres fases principales:

1.-) La fase de inicializaci\'{o}n de los par\'{a}metros $W$ y $b$.

se tiene la siguiente matriz de pesos:

\begin{center}
\qquad$W=$\bigskip$\left[
\begin{array}
[c]{cccc}%
w_{11} & w_{12} & \cdots & w_{1m}\\
w_{21} & w_{22} & \cdots & w_{2n}\\
\vdots & \vdots & \ddots & \vdots\\
w_{n1} & w_{n2} & \cdots & w_{nm}%
\end{array}
\right]  $
\end{center}

donde:

\bigskip$\qquad\qquad n:$ es el n\'{u}mero de se\~{n}ales que llegan a una
capa dada

$\qquad\qquad m:$ es el n\'{u}mero de neuronas de la capa reciben a las
se\~{n}ales ponderadas por$\ \ \ \ \ w_{i,j}$

Los valores de $w_{i,j}$ pueden ser inicializados por valores reales entre -1
y 1. El sub \'{\i}ndice $i$ indica el n\'{u}mero se se\~{n}al de donde
proviene y $j$ a que neurona se dirije.

Cabe mencionar que en el algor\'{\i}tmo aqu\'{\i} propuesto, la matriz W es
una hipermatriz donde cada capa tiene asociada la relacion de pesos entre una
capa y la subsecuente.

Como a cada neurona se le tiene que sumar un bias, entonces se puede manejar
una matriz de Bias que se enumera de arriba hacia abajo en determinada capa, como:

\begin{center}
$\qquad B=\left[
\begin{array}
[c]{ccc}%
b_{11} & b_{12} & b_{1m}\\
b_{21} & b_{22} & b_{2n}\\
\vdots & \vdots & \vdots\\
b_{nni} & b_{nnh} & b_{nn0}%
\end{array}
\right]  $
\end{center}

Donde cada columna es el arreglo de neuronas de una capa $i$ \ y cada capa
puede cambiar el n\'{u}mero de neuronas. Los valores para inicializar a la
matriz tambi\'{e}n pueden ser numeros aleatorios en un intervalo de 0 a 1.

2.-) Segunda fase, propagaci\'{o}n hacia adelante.

Como se explic\'{o} a la red se lepresentan valores de entrenamiento y los
valores que se esperan a la salida para cada una de estas entradas, entonces
se pueden representar tales entradas como vectores que poseen alg\'{u}n
patr\'{o}n que se desea identificar con un valor esperado o deseado.

Como se explic\'{o} a la red se lepresentan valores de entrenamiento y los
valores que se esperan a la salida para cada una de estas entradas, entonces
se pueden representar tales entradas como vectores que poseen alg\'{u}n
patr\'{o}n que se desea identificar con un valor esperado o deseado.

Entonces la entrada a la red puede ser escrita mediante:

$\qquad\qquad\qquad$Patr\'{o}n\ de entrada =$\left[
\begin{array}
[c]{cccc}%
x_{1} & x_{2} & \cdots & x_{n}%
\end{array}
\right]  _{i}^{T}$

\ \ \ \ El sub \'{\i}ndice $i$ indica que solo es un patr\'{o}n de todos los
que se le presentaran a la red.

Los valores esperados se escriben como:

\bigskip$\qquad\qquad\qquad$valor\ esperado=$\left[
\begin{array}
[c]{cccc}%
d_{11} & d_{12} & \cdots & d_{1n}\\
d_{21} & d_{22} & \cdots & d_{2n}\\
\vdots & \vdots & \ddots & \vdots\\
d_{kn} & d_{k2} & \cdots & d_{km}%
\end{array}
\right]  $

Donde cada rengl\'{o}n es un vector esperado para cada patr\'{o}n de entrada

\bigskip

Se procede a calcular la la salida para cada neurona i de \bigskip una capa k,
como sigue:

\bigskip$\qquad\qquad N=\left[
\begin{array}
[c]{c}%
n_{11}\\
n_{21}\\
\vdots\\
n_{nn}%
\end{array}
\right]  _{k}=\left[
\begin{array}
[c]{c}%
b_{11}\\
b_{21}\\
\vdots\\
b_{nn}%
\end{array}
\right]  _{k}+\left[
\begin{array}
[c]{cccc}%
w_{11} & w_{12} & \cdots & w_{1m}\\
w_{21} & w_{22} & \cdots & w_{2n}\\
\vdots & \vdots & \ddots & \vdots\\
w_{n1} & w_{n2} & \cdots & w_{nm}%
\end{array}
\right]  _{k}\left[
\begin{array}
[c]{c}%
x_{1}\\
x_{2}\\
\vdots\\
x_{n}%
\end{array}
\right]  _{k}$

Donde el sub \'{\i}ndice k indica el n\'{u}mero de capa en el que se desea
calcular las salidas de las neronas una vez que hayan pasado por su
funci\'{o}n de activaci\'{o}n correspondiente, entonces: la salida de la capa
k es:

$\qquad\qquad\qquad\qquad O_{k}=\varphi(N_{k})$

Si el algoritmo se encuentra en la \'{u}ltima capa se calcula el error dado
por la resta del valor esperado menos el valor en la \'{u}ltima capa.

$\qquad\qquad\qquad\qquad e=O_{last}-D$

La segunda fase es llamada de propagaci\'{o}n hacia atr\'{a}s en donde utiliza
el error calculado en la primera fase y obtienen valores delta\ o gradientes
para cada neurona desde la \'{u}ltima capa hasta la primera, dada por las
siguientes expresiones:

Para la \'{u}ltima capa:

$\qquad\qquad\qquad\qquad\delta_{last}=\varphi(B+W_{last}O_{last}%
)(1-\varphi(B+W_{last}O_{last}))e$

\bigskip Para cualquier capa escondida:

$\qquad\qquad\qquad\qquad\delta_{k}=\varphi(O_{k})(1-\varphi(O_{k}%
))\delta_{k+1}$

La tercer fase es la actualizaci\'{o}n de los valores de los pesos y las bias
de las neuronas dadas por las siguientes expresiones:

$\qquad\qquad\qquad\qquad w(n+1)=w(n)+\alpha w(n-1)+\eta\delta(n)x$

donde:

$w(n+1)$: Es la actualizaci\'{o}n de los pesos

$w(n)$: Es el peso actual

$\alpha$: Es el factor de aprendizaje

$\eta$: Factor momento

$\delta(n)$: Es el gradiente o delta en esa neurona

$\ x$: Es la entrada correspondiente\ a la neurona

\bigskip La actualizaci\'{o}n de las bias se realiza de la siguiente manera:

$\qquad\qquad\qquad\qquad\qquad b(n+1)=b(n)+\alpha b(n-1)+\eta\delta(n) $

El algor\'{\i}tmo toma todos los patrones y despu\'{e}s hace la
actualizaci\'{o}n de los pesos y las bias en funci\'{o}n del error y las
deltas. Entonces se realiza este proceso hasta que el error vaya convergiendo
en un valor muy peque\~{n}o; a esta repetici\'{o}n se le conoce como \'{e}poca
de entrenamiento.

Dependiendo del factor de entrenamiento, el momento, el n\'{u}mero de capas,
la forma de los patrones y el n\'{u}mero de clases a reconocer son las
\'{e}pocas necesarias para que el error converja en m\'{a}s o menos tiempo.

\subsubsection{Acondicionamiento de datos.}

Los patrones de entrenamiento fueron obtenidos a trav\'{e}s de Audacity, que
es un software libre para edici\'{o}n profesional de audio. Tal y como se
muestra en la siguiente figura \textit{Audacity} analiza la se\~{n}al de audio
en el dominio de la frecuencia y da como amplitud los decibeles en cada
frecuencia dentro del rango audible.

\begin{center}
\qquad\qquad\qquad%
%TCIMACRO{\FRAME{itbphF}{2.7034in}{2.041in}{0in}{}{}{2.png}%
%{\special{ language "Scientific Word";  type "GRAPHIC";
%maintain-aspect-ratio TRUE;  display "USEDEF";  valid_file "F";
%width 2.7034in;  height 2.041in;  depth 0in;  original-width 4.0802in;
%original-height 3.0727in;  cropleft "0";  croptop "1";  cropright "1";
%cropbottom "0";
%filename '../REPORTE PARTE 1/Milcom/2.png';file-properties "NPEU";}}}%
%BeginExpansion
{\includegraphics[
natheight=3.072700in,
natwidth=4.080200in,
height=2.041in,
width=2.7034in
]%
{imagenes/5-acustico/2.png}%
}%
%EndExpansion


\qquad\qquad\qquad Figura 9. Espectro analizado por \textit{Adacity}
\end{center}

Los datos fueron almacenados en hojas de c\'{a}lculo para despu\'{e}s ser
tratados y acondicionados; este proceso tuvo que ser manual debido a que el
software \textit{Audacity} exporta los datos en archivos de formato
\textit{txt.}

\qquad

\begin{center}%
%TCIMACRO{\FRAME{itbpF}{3.9098in}{2.0903in}{0in}{}{}{based.bmp}%
%{\special{ language "Scientific Word";  type "GRAPHIC";
%maintain-aspect-ratio TRUE;  display "USEDEF";  valid_file "F";
%width 3.9098in;  height 2.0903in;  depth 0in;  original-width 6.1938in;
%original-height 3.2863in;  cropleft "0";  croptop "1";  cropright "1";
%cropbottom "0";
%filename '../REPORTE PARTE 1/Milcom/baseD.bmp';file-properties "NPEU";}}}%
%BeginExpansion
{\includegraphics[
natheight=3.286300in,
natwidth=6.193800in,
height=2.0903in,
width=3.9098in
]%
{imagenes/5-acustico/baseD.png}%
}%
%EndExpansion


Figura 10. Organizaci\'{o}n de los datos en hojas de c\'{a}lculo
\end{center}

\subsubsection{Entrenamiento de la red neuronal.}

Datos de entrenamiento.

\bigskip%
%TCIMACRO{\FRAME{dtbphF}{4.3431in}{2.0678in}{0pt}{}{}{Figure}%
%{\special{ language "Scientific Word";  type "GRAPHIC";
%maintain-aspect-ratio TRUE;  display "USEDEF";  valid_file "T";
%width 4.3431in;  height 2.0678in;  depth 0pt;  original-width 11.8436in;
%original-height 5.6144in;  cropleft "0";  croptop "1";  cropright "1";
%cropbottom "0";  tempfilename 'NGLTU009.wmf';tempfile-properties "XPR";}}}%
%BeginExpansion
\begin{center}
\includegraphics[
natheight=5.614400in,
natwidth=11.843600in,
height=2.0678in,
width=4.3431in
]%
{imagenes/5-acustico/pizza.jpg}%
\end{center}
%EndExpansion


\begin{center}
Figura 11. Grafica en Matlab de los datos de entrenamiento
\end{center}

Debido a que la red neuronal est\'{a} dise\~{n}ada para trabajar con funciones
de activaci\'{o}n que se encuentran en un intervalo de 0 a 1 , los datos se
normalizaron y se les aplico una simple operaci\'{o}n matem\'{a}tica, para
invertir los datos tal y como se describe en la siguiente figura:

\begin{center}%
%TCIMACRO{\FRAME{dtbphF}{4.1528in}{3.2863in}{0pt}{}{}{Figure}%
%{\special{ language "Scientific Word";  type "GRAPHIC";
%maintain-aspect-ratio TRUE;  display "USEDEF";  valid_file "T";
%width 4.1528in;  height 3.2863in;  depth 0pt;  original-width 7.7392in;
%original-height 6.1142in;  cropleft "0";  croptop "1";  cropright "1";
%cropbottom "0";  tempfilename 'NGLTU00A.wmf';tempfile-properties "XPR";}}}%
%BeginExpansion
\begin{center}
\includegraphics[
natheight=6.114200in,
natwidth=7.739200in,
height=3.2863in,
width=4.1528in
]%
{imagenes/5-acustico/datos2.png}%
\end{center}
%EndExpansion
\qquad\qquad\qquad\qquad\qquad

Figura 12. Patrones de entrenamiento\qquad
\end{center}

Despu\'{e}s del entrenamiento, se obtuvo la hipermatriz matriz de pesos y la
matriz de bias perteneciente a cada capa. Como se muestra en la figura
siguiente, la matriz de pesos consta de tres capas de matrices de nxn, donde n
es el n\'{u}mero de datos y la matriz de bias es una matriz de una capa de
tama\~{n}o y igual al n\'{u}mero de datos del patr\'{o}n por tres columnas.

Se considera que despu\'{e}s del entrenamiento, un rasgo caracter\'{\i}stico
es la forma geom\'{e}trica de dichas matrices en el espacio, debido a que para
diferentes patrones presentados en la misma red se generar\'{a}n diferentes
valores para los pesos y las bias correspondientes un modelo grafico de la red
para discriminar diferentes patrones esta dado como se ve a continuaci\'{o}n.

\begin{center}%
%TCIMACRO{\FRAME{itbphF}{4.5671in}{2.7968in}{0in}{}{}{4.bmp}%
%{\special{ language "Scientific Word";  type "GRAPHIC";  display "USEDEF";
%valid_file "F";  width 4.5671in;  height 2.7968in;  depth 0in;
%original-width 9.1065in;  original-height 4.3405in;  cropleft "0";
%croptop "1";  cropright "1";  cropbottom "0";
%filename '../REPORTE PARTE 1/Milcom/4.bmp';file-properties "NPEU";}}}%
%BeginExpansion
{\includegraphics[
natheight=4.340500in,
natwidth=9.106500in,
height=2.7968in,
width=4.5671in
]%
{imagenes/5-acustico/4.png}%
}%
%EndExpansion


Figura 13. Hipermatriz de capas de la red neuronal
\end{center}

\bigskip

La matriz de Bias es la matriz que pondera la a cada neurona brindandole un
grado de libertad para discriminar por el nivel de la entrada y as\'{\i} ser
flexible a los cambios de nivel.

\begin{center}%
%TCIMACRO{\FRAME{dtbphF}{4.4521in}{2.0366in}{0pt}{}{}{Figure}%
%{\special{ language "Scientific Word";  type "GRAPHIC";
%maintain-aspect-ratio TRUE;  display "USEDEF";  valid_file "T";
%width 4.4521in;  height 2.0366in;  depth 0pt;  original-width 10.4063in;
%original-height 4.7392in;  cropleft "0";  croptop "1";  cropright "1";
%cropbottom "0";  tempfilename 'NGLTU00B.wmf';tempfile-properties "XPR";}}}%
%BeginExpansion
\begin{center}
\includegraphics[
natheight=4.739200in,
natwidth=10.406300in,
height=2.0366in,
width=4.4521in
]%
{imagenes/5-acustico/BIAS.png}%
\end{center}
%EndExpansion


\qquad Figura 14. Matriz de Bias de la red neuronal
\end{center}

\bigskip

\textbf{\qquad}La matriz tiene una dimensi\'{o}n igual al n\'{u}mero de datos
de entrada para los renglones y 3 columnas que indican el n\'{u}mero de capas.

Una vez que se tiene el entrenamiento de la red terminada se procede a ver el
desempe\~{n}o de la red present\'{a}ndole sonidos diferentes a los datos de
entrenamiento y verificar si dicho entrenamiento fue exitoso.Todo esto con el
proposito de saber cuando detener el entrenamiento de la red, ya que podrian
ser modificados algunos pr\'{a}metros de tal manera que la salida de la red
arrojara resultados diferenciables gr\'{a}ficamente, tal y como se muestra en
las siguientes figuras, depende el n\'{u}mero de \'{e}pocas de entrenamiento:

\begin{center}%
%TCIMACRO{\FRAME{itbpF}{4.8888in}{2.3428in}{0in}{}{}{salida5.bmp}%
%{\special{ language "Scientific Word";  type "GRAPHIC";
%maintain-aspect-ratio TRUE;  display "USEDEF";  valid_file "F";
%width 4.8888in;  height 2.3428in;  depth 0in;  original-width 9.0001in;
%original-height 4.3007in;  cropleft "0.054215";  croptop "0.99051";
%cropright "1.054215";  cropbottom "-0.00949";
%filename '../REPORTE PARTE 1/Milcom/salida5.bmp';file-properties "NPEU";}}}%
%BeginExpansion
{\includegraphics[
trim=0.487940in -0.040814in -0.487940in 0.040814in,
natheight=4.300700in,
natwidth=9.000100in,
height=2.3428in,
width=4.8888in
]%
{imagenes/5-acustico/salida5.png}%
}%
%EndExpansion


Figura 15. Resultado para 500 \'{e}pocas\qquad

\textbf{\qquad}%
%TCIMACRO{\FRAME{itbphF}{4.8161in}{2.3082in}{0in}{}{}{s6.bmp}%
%{\special{ language "Scientific Word";  type "GRAPHIC";  display "USEDEF";
%valid_file "F";  width 4.8161in;  height 2.3082in;  depth 0in;
%original-width 9.0001in;  original-height 4.3007in;  cropleft "0";
%croptop "1";  cropright "1";  cropbottom "0";
%filename '../REPORTE PARTE 1/Milcom/s6.bmp';file-properties "NPEU";}}}%
%BeginExpansion
{\includegraphics[
natheight=4.300700in,
natwidth=9.000100in,
height=2.3082in,
width=4.8161in
]%
{imagenes/5-acustico/s6.png}%
}%
%EndExpansion


Figura 16. Resultado para 600 \'{e}pocas%

%TCIMACRO{\FRAME{itbpF}{4.9199in}{2.3583in}{0in}{}{}{s8.bmp}%
%{\special{ language "Scientific Word";  type "GRAPHIC";
%maintain-aspect-ratio TRUE;  display "USEDEF";  valid_file "F";
%width 4.9199in;  height 2.3583in;  depth 0in;  original-width 9.0001in;
%original-height 4.3007in;  cropleft "0";  croptop "1";  cropright "1";
%cropbottom "0";
%filename '../REPORTE PARTE 1/Milcom/S8.bmp';file-properties "NPEU";}}}%
%BeginExpansion
{\includegraphics[
natheight=4.300700in,
natwidth=9.000100in,
height=2.3583in,
width=4.9199in
]%
{imagenes/5-acustico/S8.png}%
}%
%EndExpansion


Figura 17. Resultado para 800 \'{e}pocas
\end{center}

Como se puede observar, un sobre entrenamiento comienza generar que el entorno
semiperturbado y el natural sean para la red indistintos, entonces de acuerdo
a los experimentos realizados con la red, se escogieron valores para los
cuales esta presenta una mejor discriminaci\'{o}n de los entornos, m\'{a}s de
500 \'{e}pocas y menos de 1000.

Una vez obtenida la discriminaci\'{o}n, se lleva a cabo una correlaci\'{o}n
con una base de datos simplificada de la OMS, esta tabla posee la
informaci\'{o}n de niveles de volumen para sonido en decibeles, y ejemplos de
ambientes que se pueden parecer al presentado a la red.

\qquad

\begin{center}%
\begin{tabular}
[c]{|l|l|l|}\hline
& FUENTES DE SONIDO & dB\\\hline
1 & Umbral de audici\'{o}n & 0\\\hline
2 & Susurro, respiraci\'{o}n normal, pisadas suaves & 10\\\hline
3 & Rumor de las hojas en el campo al aire libre & 20\\\hline
4 & Murmullo, oleaje suave en la costa & 30\\\hline
5 & Biblioteca, habitaci\'{o}n en silencio, Despacho & 40\\\hline
6 & Tr\'{a}fico ligero, conversaci\'{o}n normal & 50\\\hline
7 & Oficina grande en horario de trabajo & 60\\\hline
8 & Conversaci\'{o}n en voz muy alta, griter\'{\i}a, tr\'{a}fico,intenso de
ciudad & 70\\\hline
9 & Timbre, cami\'{o}n pesado movi\'{e}ndose & 80\\\hline
10 & Aspiradora funcionando, maquinaria de una f\'{a}brica trabajando &
90\\\hline
11 & Banda de m\'{u}sica rock, Motocicleta & 100\\\hline
12 & Claxon de un coche, explosi\'{o}n de petardos & 110\\\hline
13 & Umbral del dolor & 120\\\hline
14 & Martillo neum\'{a}tico (de aire) & 130\\\hline
15 & Avi\'{o}n de reacci\'{o}n durante el despegue & 150\\\hline
16 & Motor de un cohete espacial durante el despegue & 180\\\hline
\end{tabular}


Tabla 2. Rangos de sonido de la OMS \ [3]
\end{center}

Cabe mencionar que todos los datos que forman parte de la informaci\'{o}n que
necesita este sistema clasificador fueron vaciados en hojas de c\'{a}lculo
para su f\'{a}cil manejo computacional.\

\ El resultado del conjunto de facultades de este sistema calificador, es
mostrar el espectro analizado a manera de comprobaci\'{o}n del
performance\bigskip de la red neuronal, ejemplos asociados a los intervalos de
volumenes y como la red identifica el sonido analizado, todo esto en una
aplicaci\'{o}n de escritorio \textit{standalone.}

En la siguiente figura se muestra la aplicaci\'{o}n realizada en un formulario
de la paqueter\'{\i}a \textit{Matlab }; como se observa, la interfaz cuenta
con un comando que abre el explorador de archivos para escoger la hoja de
c\'{a}lculo, despu\'{e}s se realiza la calificaci\'{o}n del entorno y un
ejemplo de qu\'{e} provocar\'{\i}a dicho espectro seg\'{u}n la OMS.

\newpage

\begin{center}
\qquad\qquad%
%TCIMACRO{\FRAME{itbpF}{2.8513in}{2.2451in}{0in}{}{}{aplicaci�n2.bmp}%
%{\special{ language "Scientific Word";  type "GRAPHIC";
%maintain-aspect-ratio TRUE;  display "USEDEF";  valid_file "F";
%width 2.8513in;  height 2.2451in;  depth 0in;  original-width 7.1866in;
%original-height 5.6394in;  cropleft "0";  croptop "1";  cropright "1";
%cropbottom "0";
%filename '../REPORTE PARTE 1/Milcom/aplicaci�n2.bmp';file-properties "NPEU";}%
%}}%
%BeginExpansion
{\includegraphics[
natheight=5.639400in,
natwidth=7.186600in,
height=2.2451in,
width=2.8513in
]%
{imagenes/5-acustico/aplicacion2.png}%
}%
%EndExpansion
\qquad\qquad

Figura 18. Aplicaci\'{o}n standalone
\end{center}

\subsection{Modo de operaci\'{o}n de la aplicaci\'{o}n.}

Esta versi\'{o}n b\'{a}sica de la aplicaci\'{o}n consta del manejo del
software \textit{Audacity}, una hoja de c\'{a}lculo y la aplicaci\'{o}n de
escritorio dise\~{n}ada en \textit{Matlab}, como se observa en la figura el
usuario necesita analizar con \textit{Audacity} el sonido en formato wav, que
previamente ha guardado en el ordenador, despu\'{e}s de la sesi\'{o}n de grabaci\'{o}n.

En el menu abrir se selecciona el archivo wav y una vez cargado, seguir los
siguientes pasos:

Menu analizar, escoger formato espectro y dar clic en el comando exportar, la
exportaci\'{o}n de los datos del espectro es en formato txt, ese archivo se
guardar\'{a} en la misma carpeta donde se encuentra el archivo wav, una vez
que se tienen los datos en txt se debe copiar manualmente los vectores y
ponerlos en hojas de calculo para su f\'{a}cil manejo en el procesamiento de
la aplicaci\'{o}n de escritorio.

Entonces la aplicaci\'{o}n solicita al usuario identificar la columna en la
cual estan almacenados los datos, y el n\'{u}mero de hoja en el que se encuentran.

Como se aprecia en la figura 18 (\textit{vease figura anterior}) el resultado
es la califacci\'{o}n entre las clases \textit{PERTURBADO,}
\textit{SEMI-PERTURBADO} y \textit{NATURAL, }adem\'{a}s se despliega una
situaci\'{o}n en la que es an\'{a}logo el volumen analizado con respecto a la
tabla de la OMS.

\section*{Conclusiones.}

El desempe\~{n}o de la Red Neuronal no solo depende de los par\'{a}metros
$\alpha,$ \ y $\eta,$ , si no tambi\'{e}n del n\'{u}mero de \'{e}pocas y la
forma de los datos de entrenamiento, para esto fue necesario del uso de
softwares adicionales, para que el uso de los datos en otro dominio generara
informaci\'{o}n razonablemente.

La calidad de identificaci\'{o}n depende en gran medida de los instrumentos
utilizados, debido a que por ejemplo, si se quisiera llevar todo el
procesamiento directamente a un dispositivo m\'{o}vil, se estar\'{\i}a
dependiendo de la calidad de grabaci\'{o}n que posee, aspecto que en la
actualidad no es tiene buena calidad al compararlo con grabadoras
profesionales, entonces, se puede decir que hasta que se evolucione la
tecnolog\'{\i}a en cuanto a la grabaci\'{o}n de audio, ser\'{a} posible
implementar la aplicaci\'{o}n presentada en los dispositivos m\'{o}viles.

En cuanto a paisajes sonoros se refiere los estudios sugieren en un momento
que la contaminaci\'{o}n ac\'{u}stica se ha convertido en un problema
reconocido y ampliamente extendido, cualesquiera sean las acciones que se han
tomado hasta ahora para mitigar este problema parecen no tener suficiente
efecto. Hasta este punto, en el presente proyecto, se ha podido obtener una
m\'{e}trica de la calidad de los paisajes sonoros, lo que sigue sera
desarrollar herramental de mejor\'{\i}a del nivel de bienestar psicol\'{o}gico
a trav\'{e}s de la s\'{\i}ntesis de paisajes ac\'{u}sticos de calidad.

\section*{Referencias.}

$^{1}$. La Bauhaus fue una escuela de arte, arquitectura y dise\~{n}o; fundada
por el arquitecto, urbanista y dise\~{n}ador de origen germano Walter Gropius
en Weimar ( Alemania) en 1919.

$^{2}$. Truax, Acoustic Communications, 2001, pp 11.

$^{3}$. Moles, La imagen sonora del mundo circulante, fonograf\'{\i}as y
paisajes sonoros 1999, pp 229-243.

$^{4}$. Schafer sostiene que en algunos casos estos sonidos podr\'{\i}an ser
arquet\'{\i}picos. Estar en el subconsciente del escucha ya que han sido
percibidos durante milenios por sus antepasados.

$^{5}$. Moles describe los elementos figura y fondo en oposici\'{o}n, como
elementos constitutivos del paisaje sonoro. (Moles, La imagen sonora del mundo
circulante, fonograf\'{\i}as y paisajes sonoros 1999, pp 229-243)

$^{6}$. Esto ser\'{\i}a la relaci\'{o}n entre el primero y segundo planos, es
decir, entre las figuras y el fondo, descritos por Schafer.

$^{7}$. Manuel Rocha Iturbide interpreta perspectiva como amplitud de fondo,
es decir, en donde varias capas s\'{o}nicas se pueden manifestar y
distinguirse unas de otras.

$^{8}$. Schafer, The tuning of the World, 1977, pp43.

\section*{Bibliografia consultada.}

[1] Atienza Ricardo, Ambientes sonoros urbanos : la identidad sonora. Modos de
Permanencia y Variaci\'{o}n de una configuraci\'{o}n urbana, 2007, p\'{a}g
2,3, 6,8 y 13.

[2] Jos\'{e} Luis CARLES / Cristina PALMESE (2004): Identidad sonora urbana. http://www.eumus.edu.uy

[3] Robert MURRAY SCHAFER (1977) : The tuning of the world. Toronto. Ed.
McClelland and Steward.

[4] Hildegard Westerkamp. Brahaus y estudios sobre el paisaje sonoro -
Explorando conexiones y diferencias: El surgimiento de los estudios sobre el
paisaje sonoro.

[5] Max Nehaus. Dise\~{n}o sonoro.

[6] Hans-Ulrich Werner. Tres instant\'{a}neas sobre el paisaje sonoro.
Dise\~{n}o ac\'{u}stico - Dise\~{n}o sonoro.

[7] Sol Rezza. Revista de pensament musical, n\'{u}m. 004: El mundo es un
paisaje sonoro (3 percepciones respecto al paisaje sonoro).

[8] Barry Truax: Compositor canadiense especializado en las implementaciones
en tiempo real de la s\'{\i}ntesis granular, a menudo de sonidos grabados y
paisajes sonoros. En 1986 desarroll\'{o} la primera implementaci\'{o}n en
tiempo real de la s\'{\i}ntesis granular. Hizo la primera obra en
s\'{\i}ntesis granular hecha a base en sonidos pregrabados, Wings of Nike en
1987 y fue el primer compositor en explorar el \'{a}rea entre la s\'{\i}ntesis
granular sincr\'{o}nica y asincr\'{o}nica en Riverrun, en1986. La t\'{e}cnica
en tiempo real sigue o enfatiza en las corrientes auditor\'{\i}as, que junto
con los paisajes sonoros hacen parte importante de su est\'{e}tica. Truax
ense\~{n}a m\'{u}sica electroac\'{u}stica, m\'{u}sica por computador y
comunicaci\'{o}n ac\'{u}stica en la Universidad Simon Fraser en Canad\'{a}.
Fue uno de los miembros fundadores junto con Raymond Murray Schaffer del World
Soundscape Project.

[9] Jordi Pigem. Escuchar las voces del mundo.

[10] Fundamentals of Neural Networks, architectures, Algorithms and
applications, Laurene Fausett , 1994, Prentice Hall.\

[11] http://www.fceia.unr.edu.ar/acustica/biblio/omscrit.htm

\bigskip

\qquad

\qquad

\begin{center}
\bigskip
\end{center}


% \end{document}
